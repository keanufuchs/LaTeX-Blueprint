%% ++++++++++++++++++++++++++++++++++++++++++++++++++++++++++++
%% Hauptdatei, Wurzel des Dokuments
%% ++++++++++++++++++++++++++++++++++++++++++++++++++++++++++++
%
%  Gerüst:
%  * Version 0.13
%  * M.Sc. Silvia Krug, silvia.krug@tu-ilmenau.de
%  * Fachgebiet Kommunikationsnetze, TU Ilmenau
%
%  Für Hauptseminare, Studienarbeiten, Diplomarbeiten
%
%  Autor           : Max Mustermann
%  Letzte Änderung : 04.12.2015
%

% Headerfeld, Typ des Dokumentes, einzubindende Packages.
% Hier bei Bedarf Änderungen vornehmen.
\documentclass
[   twoside=false,     % Einseitiger oder zweiseitiger Druck?
    fontsize=12pt,     % Bezug: 12-Punkt Schriftgröße
    DIV=15,            % Randaufteilung, siehe Dokumentation "KOMA"-Script
    BCOR=17mm,         % Bindekorrektur: Innen 17mm Platz lassen. Copyshop-getestet.
%    headsepline,
    headsepline,  % Unter Kopfzeile Trennlinie (aus: headnosepline)
    footsepline,  % Über Fußzeile Trennlinie (aus: footnosepline)
  open=any,          % Kapitel/Verzeichnisse nicht auf rechte Seite erzwingen
    paper=a4,          % Seitenformat A4
    abstract=true,     % Abstract einbinden
    listof=totoc,      % Div. Verzeichnisse ins Inhaltsverzeichnis aufnehmen
    bibliography=totoc,% Literaturverzeichnis ins Inhaltsverzeichnis aufnehmen
    titlepage,         % Titelseite aktivieren
    headinclude=true,  % Seiten-Head in die Satzspiegelberechnung mit einbeziehen
    footinclude=false, % Seiten-Foot nicht in die Satzspiegelberechnung mit einbeziehen
    numbers=noenddot   % Gliederungsnummern ohne abschließenden Punkt darstellen
]   {scrreprt}         % Dokumentenstil: "Report" aus dem KOMA-Skript-Paket


\usepackage{makecell} % Im Header Ihrer LaTeX-Datei einfügen
\usepackage{slantsc}
\usepackage{caption}
% Caption-Package so konfigurieren, dass vollständige Nummerierung in Verzeichnissen verwendet wird
\captionsetup{listformat=simple,font=small,labelfont=bf}
% Wissenschaftlicher Standard:
% - Abbildungsunterschriften unter der Abbildung
% - Tabellenüberschriften über der Tabelle
\captionsetup[figure]{position=bottom,justification=raggedright,singlelinecheck=false}
\captionsetup[table]{position=top,justification=raggedright,singlelinecheck=false}
\usepackage{multirow}
\usepackage[active]{srcltx}
%\usepackage[activate=normal]{pdfcprot} % Optischer Randausgleich -> pdflatex!
\usepackage{ifthen}
\usepackage[ngerman]{babel}   % Neue Deutsche Rechtschreibung
\usepackage{iftex}
\ifPDFTeX
%\usepackage[latin1]{inputenc} % Zeichencodierung nach ISO-8859-1
\usepackage[utf8]{inputenc}   % Zeichencodierung nach UTF-8 (Unicode)
\usepackage[T1]{fontenc}
%\usepackage{ae} % obsolet und durch lmodern ersetzt
\usepackage{lmodern}
\else
\usepackage{fontspec}
\fi
\usepackage{microtype} % Bessere Silbentrennung, Zeichenprotrusion, weniger Overfull/Underfull
\usepackage{parskip}   % Absätze durch Abstand statt Einrückung trennen (wie in Word)
\usepackage[T1]{url}
\usepackage[final]{graphicx}
\usepackage{tikz}
\usepackage{pgfgantt}
\usetikzlibrary{arrows.meta,shapes.geometric}
\usepackage{float}         % Option H für exakte Float-Positionierung
\floatplacement{figure}{H} % Standard: Abbildungen an der definierten Stelle setzen
\floatplacement{table}{H}  % Standard: Tabellen an der definierten Stelle setzen
\usepackage[section]{placeins} % Verhindert, dass Floats die Sektion verlassen
\usepackage{needspace}     % Reserviert Platz, um Überschrift + Inhalt zusammenzuhalten
\usepackage[automark]{scrlayer-scrpage}
\usepackage{setspace}
\usepackage[printonlyused]{acronym}
%\usepackage[first,light]{draftcopy} % Für Probedruck
\usepackage[plainpages=false,pdfpagelabels,hypertexnames=false]{hyperref}

\usepackage[ngerman]{babel}  % Für deutsche Spracheinstellungen
\usepackage{datetime}        % Für Datumsformatierung

% Definiere ein neues Datumsformat
\newdateformat{german}{\twodigit{\THEDAY}.\twodigit{\THEMONTH}.\THEYEAR}
\usepackage{url}  % For typesetting URLs if required
\usepackage{xurl} % Besserer Zeilenumbruch für lange URLs und Pfade
\usepackage{siunitx} % Einheitliche Zahlen- und Einheitenformatierung
\sisetup{
  locale=DE,
  output-decimal-marker={,},
  group-separator={.},
  group-minimum-digits=4
}

\usepackage{verbatim} % Ermöglicht die Verwendung der verbatim-Umgebung
%\usepackage{fontspec}
\usepackage[htt]{hyphenat} % Erlaubt Silbentrennung in \texttt{} (Typewriter)

% Zitate, Quotes, Cite
\usepackage[backend=bibtex, sorting=nyt, style=ieee-alphabetic, minalphanames=1, maxbibnames=99]{biblatex}
\addbibresource{bib/literatur.bib}  % Pfad zur Literaturdatenbank

\DeclareLabelalphaTemplate{
  \labelelement{
    \field[final]{shorthand}
    \field{label}
    \field[strwidth=3,strside=left]{labelname}
  }
  \labelelement{
    \field[strwidth=2,strside=right]{year}    
  }
}
\usepackage{csquotes}

\usepackage{listings}
\usepackage{color}

\definecolor{dkgreen}{rgb}{0,0.6,0}
\definecolor{gray}{rgb}{0.5,0.5,0.5}
\definecolor{mauve}{rgb}{0.58,0,0.82}

\lstset{
  frame=tb,
  language=Python,
  aboveskip=3mm,
  belowskip=3mm,
  showstringspaces=false,
  columns=flexible,
  basicstyle={\ttfamily\small},
  numbers=left,
  numberstyle=\tiny\color{gray},
  keywordstyle=\color{blue},
  commentstyle=\color{dkgreen},
  stringstyle=\color{mauve},
  breaklines=true,
  breakatwhitespace=true,
  tabsize=3,
  captionpos=b,      % Position der Beschriftung (bottom)
  escapeinside={(*@}{@*)}, % Ermöglicht LaTeX-Befehle innerhalb von Code
  literate={ä}{{\"a}}1 {ö}{{\"o}}1 {ü}{{\"u}}1 {Ä}{{\"A}}1 {Ö}{{\"O}}1 {Ü}{{\"U}}1 {ß}{{\ss}}1
}

% Umbenennung von "Listing" zu "Codebeispiel"
\renewcommand{\lstlistingname}{Codebeispiel}
\renewcommand{\lstlistlistingname}{Codebeispielverzeichnis}

% Tiefe der Kapitelnummerierung beeinflussen
\setcounter{secnumdepth}{3} % Tiefe der Nummerierung
\setcounter{tocdepth}{3}    % Tiefe des Inhaltsverzeichnisses

% Eigene Befehle
\newcommand{\uproman}[1]{\uppercase\expandafter{\romannumeral#1}}
% Quellenangaben unter Abbildungen/Tabellen (wissenschaftlicher Standard)

% Vermeidet Schusterjungen/Hurenkinder (einzelne Zeilen am Seitenanfang/-ende)
\clubpenalty=10000
\widowpenalty=10000
\displaywidowpenalty=10000

% Hier in die zweite geschweifte Klammer jeweils
% die persönlichen Daten und das Thema der Arbeit eintragen:
\newcommand{\artderausarbeitung}{Art der Ausarbeitung}
\newcommand{\namedesautors}{Max Mustermann}
\newcommand{\matrikelnummer}{1234567}
\newcommand{\themaderarbeit}{Titel der Arbeit}
\newcommand{\fachbereich}{Fachbereich}
\newcommand{\studiengang}{Studiengang}
\newcommand{\praxispartner}{Musterfirma GmbH} 
\newcommand{\ort}{Musterstadt}

% PDF Metadaten definieren  
\hypersetup{
   pdftitle={\themaderarbeit},
   pdfsubject={\artderausarbeitung},
   pdfauthor={\namedesautors},
   pdfkeywords={\artderausarbeitung; LaTeX; Vorlage;},
   colorlinks=false,       % Keine farbigen Links
   allbordercolors={0 0 0},% Alle Umrandungen schwarz
   pdfborderstyle={/S/U/W 1} % Unterstrichen (Underline) mit 1pt Breite
}

\usepackage[normalem]{ulem}
  \newcommand{\markup}[1]{\textbf{#1}}



% Seitenlayout festlegen. Hier nichts ändern!
\pagestyle{scrplain}
\ihead[]{\headmark}
\ohead[]{}
\chead[]{}
\ifoot[]{\scriptsize \artderausarbeitung\ \namedesautors}
\ofoot[]{\pagemark}
\cfoot[]{}
\renewcommand{\titlepagestyle}{scrheadings}
\renewcommand{\partpagestyle}{scrheadings}
\renewcommand{\chapterpagestyle}{scrheadings}
\renewcommand{\indexpagestyle}{scrheadings}

% Abschnittsweise Nummerierung anstatt fortlaufend. Hier nichts ändern!
\makeatletter
\@addtoreset{equation}{chapter}
\@addtoreset{figure}{chapter}
\@addtoreset{table}{chapter}
\renewcommand\theequation{\thechapter.\@arabic\c@equation}
\renewcommand\thefigure{\thechapter.\@arabic\c@figure}
\renewcommand\thetable{\thechapter.\@arabic\c@table}
\makeatother

% Formatierung für Verzeichnisse: vollständige Nummerierung anzeigen
% KOMA-Script bietet spezielle Befehle für die Verzeichnisformatierung
\DeclareTOCStyleEntry[numwidth=2.5em]{tocline}{figure}
\DeclareTOCStyleEntry[numwidth=2.5em]{tocline}{table}

% Quelltextrahmen, klein. Hier nichts ändern!
\newsavebox{\inhaltkl}
\def\rahmenkl{\sbox{\inhaltkl}\bgroup\small\renewcommand{\baselinestretch}{1}\vbox\bgroup\hsize\textwidth}
\def\endrahmenkl{\par\vskip-\lastskip\egroup\egroup\fboxsep3mm%
\framebox[\textwidth][l]{\usebox{\inhaltkl}}}

% Quelltextrahmen, normale Groesse. Hier nichts ändern!
\newsavebox{\inhalt}
\def\rahmen{\sbox{\inhalt}\bgroup\renewcommand{\baselinestretch}{1}\vbox\bgroup\hsize\textwidth}
\def\endrahmen{\par\vskip-\lastskip\egroup\egroup\fboxsep3mm%
\framebox[\textwidth][l]{\usebox{\inhalt}}}

% Trennvorschläge für falsch getrennte Wörter.
\hyphenation{
Hard-ware
Daten-verarbeitungs-prozess
Projekt-ordner-struktur
Konvertierungs-logik
Validierungs-strategie
Abbildungs-verzeichnis
Tabellen-verzeichnis
Codebeispiel-verzeichnis
}

% Sonstige Befehlsdefinitionen hier ablegen.
\newcommand{\entspricht}{\stackrel{\wedge}{=}}

% Tabellenspaltendefinitionen mit fester Breite --> somit Zeilenumbruch innerhalb einer Zelle möglich
% aus http://www.torsten-schuetze.de/tex/tabsatz-2004.pdf
\usepackage{array, booktabs}
\newcolumntype{f}{>{$}l<{$}}
\newcolumntype{n}{>{\raggedright}l}
\newcolumntype{N}{>{\scriptsize}l}
\newcolumntype{v}[1]{>{\raggedright\hspace{0pt}}m{#1}}
\newcolumntype{V}[1]{>{\scriptsize\raggedright\hspace{0pt}}m{#1}}
\newcolumntype{Z}[1]{>{\raggedright\centering}m{#1}}
\newcolumntype{k}[1]{>{\raggedright}p{#1}}
% ergibt Tabllenspalte fester Breite, linksbündig
% Umbruch innerhalb der Zelle mit \\, neue Tabellezeile mit \tabularnewline
% \addlinespace für Gruppentrennung (aus \texttt{booktabs.sty})




% Anpassung der Abstände bei Kapitelüberschriften
\RedeclareSectionCommand[
  beforeskip=-10pt, % Abstand über der Überschrift (Standard ca. 50pt)
  afterskip=20pt    % Abstand unter der Überschrift
]{chapter}

\begin{document}
% Verhindere, dass \cleardoublepage Leerseiten erzeugt (insb. bei open=any)
\makeatletter
\let\cleardoublepage\clearpage
\makeatother
% Global etwas mehr Spielraum für Umbrüche zur Reduktion von Overfull \hbox-Fehlern
\emergencystretch=2em
  
\onehalfspacing
    

\begin{titlepage}
\centering

{\Large \textsc{Rheinische Hochschule Köln}}\\
University of Applied Sciences\\[2ex]
{\large{Fachbereich: \fachbereich}}\\
{Studiengang: \studiengang}\\

% Implement a gap between the lines
\vspace{1cm}    

\includegraphics[scale=0.2]{bilder/rfh_logo.png}\\[3ex]
\vfill
{\Large \textbf{\artderausarbeitung}}\\[3ex]
{\large \textbf{\themaderarbeit}}\\[5ex]
\vfill
\begin{tabular}{rl}
\hline\\
vorgelegt von:          & \quad \namedesautors\\[1,5ex]
                        & \quad Schlesienstr. 5\\[1,5ex]
                        & \quad 53119, Bonn\\[1,5ex]
geboren am:             & \quad 07.\,03.\,2003 in Bonn\\[1,5ex]
Matrikelnummer:			& \quad \matrikelnummer\\[1,5ex]
Fachsemester:			& \quad 5\\[1,5ex]
eingereicht am:         & \quad 20.\,12.\,2025\\[1,5ex]


Dozentin:               & \quad Prof. Dr. Friedel Mager\\[1,5ex]
Praxisparnter:          & \quad \bechtlebonn\\[1,5ex]  

\end{tabular}
\vfill

Wintersemester 2025/2026
\end{titlepage}








%

\section*{Danksagung}
\ldots \emph{
Vielen Dank an Christian Drefke, meinen Vorgesetzten sowie Studienbegleiter, für seine wertvolle Unterstützung während meiner Arbeit an diesem Projekt. Christians Beiträge und Ratschläge waren von unschätzbarem Wert}\ldots


%\input{meta/kurzfassung.tex}


% Inhaltsverzeichnis
\cleardoublepage % Seitenumbruch erzwingen vor Änderung des Nummerierungsstils
\pagenumbering{Roman} % Nummerierung der Seiten ab hier: i, ii, iii, iv...
\pagestyle{scrheadings} % Ab hier mit Kopf- und Fusszeile

% Abkürzungsverzeichnis einbinden


\clearpage

\section*{Abkürzungsverzeichnis}
\ihead[]{Abkürzungsverzeichnis}
\begin{acronym}
    \acro{ACI}{Application Centric Infrastructure}
    \acro{API}{Application Programming Interface}
    \acro{BD}{Bridge Domain}
    \acro{CLI}{Command Line Interface}
    \acro{CSV}{Comma-Separated Values}
    \acro{EPG}{Endpoint Group}
    \acro{ESG}{Endpoint Security Group}
    \acro{FQDN}{Fully Qualified Domain Name}
    \acro{GUI}{Graphical User Interface}
    \acro{HTTP}{Hypertext Transfer Protocol}
    \acro{HTTPS}{Hypertext Transfer Protocol Secure}
    \acro{IaC}{Infrastructure as Code}
    \acro{IDE}{Integrated Development Environment}
    \acro{IP}{Internet Protocol}
    \acro{JSON}{JavaScript Object Notation}
    \acro{L3Out}{Layer 3 Outside Connection}
    \acro{NaC}{Network as Code}
    \acro{REST}{Representational State Transfer}
    \acro{SDK}{Software Development Kit}
    \acro{SSH}{Secure Shell}
    \acro{SSL}{Secure Sockets Layer}
    \acro{TCP}{Transmission Control Protocol}
    \acro{TLS}{Transport Layer Security}
    \acro{UDP}{User Datagram Protocol}
    \acro{UI}{User Interface}
    \acro{URL}{Uniform Resource Locator}
    \acro{UX}{User Experience}
    \acro{VLAN}{Virtual Local Area Network}
    \acro{VM}{Virtual Machine}
    \acro{VMM}{Virtual Machine Manager}
    \acro{VRF}{Virtual Routing and Forwarding}
    \acro{XLSX}{Microsoft Excel Open XML Spreadsheet}
    \acro{YAML}{Yet Another Markup Language}
    \acro{IPv4}{Internet Protocol Version 4}
    \acro{IPv6}{Internet Protocol Version 6}
\end{acronym}






\clearpage % Seitenumbruch erzwingen vor Änderung des Nummerierungsstils


\ihead[]{Inhaltsverzeichnis}
\tableofcontents


\cleardoublepage % Seitenumbruch erzwingen vor Änderung des Nummerierungsstils
% \chapter*{Hinweise}
\ihead{Hinweise}

\addcontentsline{toc}{chapter}{Hinweise} % Fügt die Überschrift in das Inhaltsverzeichnis ein, ohne zu nummerieren.

Diese Arbeit beschreibt das Projekt \textbf{XL Fabric}, ein Python-Tool zur Validierung und Konvertierung von Cisco \ac{ACI} Konfigurationsdaten. Das Tool liest Day-0-Konfigurationen aus Excel-Dokumenten ein und transformiert diese in Terraform-kompatible \acf{YAML}-Dateien. Der Schwerpunkt des Projekts liegt auf der zweistufigen Validierung mittels Pydantic: Zunächst werden die Excel-Daten auf Vollständigkeit, korrekte Formate und Konsistenz geprüft (Input-Validierung), anschließend wird die Terraform-Kompatibilität und Datenintegrität vor dem Export sichergestellt (Output-Validierung).

\textbf{Wichtiger Hinweis:} XL Fabric konzentriert sich ausschließlich auf die Terraform-Ausgabe. Obwohl im Code Ansible-Funktionalität vorbereitet ist (\acf{CLI}-Flag \texttt{--ansible}), wurde diese nicht implementiert oder weiterentwickelt. Die gesamte Dokumentation und alle Tests beziehen sich daher ausschließlich auf den Terraform-Workflow. Die tatsächliche Anwendung der Konfiguration auf die \ac{ACI} Fabric mittels Terraform, die Entwicklung oder Anpassung der verwendeten Terraform-Module sowie die Verwaltung oder Überwachung von \ac{ACI}-Infrastrukturen sind explizit nicht Teil dieses Projekts.

In dieser Arbeit werden häufig Abkürzungen verwendet. Bei der ersten Verwendung wird die vollständige Bezeichnung angegeben, gefolgt von der Abkürzung in Klammern. Ein vollständiges Abkürzungsverzeichnis findet sich zu Beginn dieser Dokumentation. Code-Beispiele werden in Verbatim-Umgebungen dargestellt und verwenden Python 3.11 Syntax. Die verwendeten Bibliotheken und ihre Versionen sind im Kapitel \enquote{Vorgehensweise und Methodik} dokumentiert. % Einleitung

% Die einzelnen Kapitel
\cleardoublepage % Seitenumbruch erzwingen vor Änderung des Nummerierungsstils
\pagenumbering{arabic} % Nummerierung der Seiten ab hier: 1, 2, 3, 4...
% Description: Kapitel 1 der Dokumentation
\ihead{Projektbeschreibung}
\chapter{Projektbeschreibung}

\section{Einführung in das Projekt XL Fabric}
Komplexe Netzwerke benötigen eine verlässliche, automatisierte Datenaufbereitung. \textbf{XL Fabric} konvertiert Excel-basierte Day-0-Konfigurationen für Cisco \acf{ACI} in validierte, Terraform-kompatible \acf{YAML}-Dateien. Das Python-Tool setzt auf Pydantic und führt eine zweistufige Validierung (Input und Output) durch, bevor die Daten exportiert werden. Es ist für die \ac{NaC} \ac{ACI}-Module ausgelegt und konzentriert sich bewusst auf Validierung und Konvertierung – nicht auf die eigentliche \ac{ACI}-Konfiguration.

\section{Ausgangssituation und Problemstellung}
Day-0-Konfigurationen werden häufig in Excel mit Parametern wie Tenants, \ac{BD}s, Contracts und Application Profiles gepflegt und bilden die Planungsgrundlage. Die manuelle Überführung in Terraform-Variablen ist jedoch fehleranfällig, langsam und intransparent: Ungültige \ac{IP}-Adressen, fehlende Pflichtfelder und inkonsistente Referenzen fallen oft erst beim Deployment auf; Copy-Paste-Fehler und uneinheitliche Konventionen erschweren Wartung und Review; große Umfänge kosten viel Zeit und die Nachvollziehbarkeit von Transformationen ist gering. Eine automatisierte, validierende Konvertierung reduziert diese Risiken deutlich.

\section{Projektziel}
Ziel ist ein kompaktes Python-Tool, das Excel-Dateien (\texttt{pandas}/\texttt{openpyxl}) einliest, die Eingabedaten mit Pydantic-Modellen validiert, sie in die Struktur der \ac{NaC} \ac{ACI}-Module überführt, das Ergebnis erneut gegen Output-Modelle prüft und schließlich als \ac{YAML} für Terraform exportiert. Pflichtfelder, Typen, Netzparameter (z. B. \ac{IP}, Subnetze, \ac{VLAN}-IDs) und Referenzen werden konsequent geprüft; die Exportstruktur ist direkt für das Deployment nutzbar. Der Fokus liegt \textbf{ausschließlich} auf Validierung und Konvertierung, nicht auf der Ausführung der Terraform-Module.

\section{Abgrenzung}
Nicht Bestandteil sind die Entwicklung oder Anpassung von Terraform-Modulen, das eigentliche Apply auf die \ac{ACI} Fabric, der Betrieb oder das Monitoring von Infrastrukturen, die Pflege der Excel-Vorlagen, Integrationen jenseits von Cisco \ac{ACI}, Backup/Restore-Funktionen, eine \ac{GUI} sowie das Management von Terraform State oder Backends. XL Fabric liefert valide Variablen – die Ausführung übernimmt die bestehende Pipeline.

\section{Projektumfeld und Einsatzszenario}
Das Tool adressiert Rechenzentrumsumgebungen auf Cisco \ac{ACI} und richtet sich an Teams, die \ac{IaC} mit Terraform nutzen. Typischer Ablauf: In der Planung entstehen Excel-basierte Day-0-Daten; XL Fabric validiert und konvertiert sie zu \ac{YAML}; Terraform plant und deployed; Dateien werden versioniert in Git verwaltet. Es integriert sich in bestehende Repositories und \ac{CI/CD} (z. B. GitLab) sowie die \ac{NaC}-Module (Namespace: netascode) und folgt deren Strukturvorgaben für eine reibungslose Übergabe.

\section{Technologische Grundlagen}
Eingesetzt werden \textbf{Python 3.11}, \textbf{Pydantic 2.x} für die Validierung, \textbf{pandas}/\textbf{openpyxl} für den Excel-Import, \textbf{PyYAML} für den Export sowie \textbf{Click} für die \acf{CLI}. Die Kombination liefert eine performante, wartbare und erweiterbare Grundlage für die \ac{ACI}-Konfigurationsverwaltung.


\ihead{LaTeX Beispiele}
\chapter{Beispiele für LaTeX-Elemente}

Dieses Kapitel demonstriert die Verwendung häufiger Elemente in wissenschaftlichen Arbeiten.

\section{Zitate und Literaturverzeichnis}
Zitate sind essentiell für wissenschaftliches Arbeiten.
\begin{itemize}
    \item Ein Buch zitieren: \cite{mustermann2023}.
    \item Einen Artikel zitieren: \cite{doe2024}.
    \item Eine Webseite zitieren: \cite{latexproject}.
\end{itemize}
Das Literaturverzeichnis wird automatisch basierend auf den zitierten Werken erstellt.

\section{Abkürzungen (Akronyme)}
Abkürzungen sollten bei der ersten Verwendung ausgeschrieben werden. Das \texttt{acronym}-Paket übernimmt dies automatisch.
\begin{itemize}
    \item Erste Verwendung: \ac{API}.
    \item Zweite Verwendung: \ac{API}.
    \item Plural: \acp{API}.
    \item Ein weiteres Beispiel: \ac{REST}.
\end{itemize}
Die Definitionen befinden sich in \texttt{meta/abkuerzungsverzeichnis.tex}.

\section{Abbildungen}
Abbildungen werden mit der \texttt{figure}-Umgebung eingebunden. Referenzieren Sie immer auf die Abbildung im Text (siehe Abbildung \ref{fig:beispiel}).

\begin{figure}[ht]
    \centering
    \includegraphics[width=0.5\textwidth]{bilder/Beispiel-Diagramm.png}
    \caption{Ein beispielhaftes Diagramm}
    \label{fig:beispiel}
\end{figure}

\section{Tabellen}
Tabellen können einfach oder komplex sein. Tabelle \ref{tab:beispiel} zeigt ein einfaches Beispiel.

\begin{table}[ht]
    \centering
    \begin{tabular}{|l|c|r|}
        \hline
        \textbf{Links} & \textbf{Zentriert} & \textbf{Rechts} \\
        \hline
        Wert A & 123 & 10,50 \\
        Wert B & 456 & 20,00 \\
        \hline
    \end{tabular}
    \caption{Eine Beispieltabelle}
    \label{tab:beispiel}
\end{table}

\section{Quellcode}
Für Quellcode eignet sich das \texttt{listings}-Paket.

\begin{lstlisting}[language=Python, caption={Ein Python-Beispiel}, label={code:python}]
def hello_world():
    print("Hello, LaTeX Template!")
\end{lstlisting}

Referenz auf das Codebeispiel: \ref{code:python}.

\section{Mermaid-Diagramm}
Mermaid-Diagramme koennen als Quelltext dokumentiert und als Diagramm dargestellt werden.

\begin{figure}[ht]
    \centering
    \begin{tikzpicture}[
        >={Stealth[length=3mm]},
        line/.style={draw=black!70, thick, ->},
        term/.style={draw=teal!60!black, fill=teal!15, rounded corners=2pt, minimum width=3.0cm, minimum height=0.9cm, align=center, font=\small},
        process/.style={draw=blue!60!black, fill=blue!10, rounded corners=2pt, minimum width=3.4cm, minimum height=0.95cm, align=center, font=\small},
        inputstep/.style={draw=violet!60!black, fill=violet!10, rounded corners=2pt, minimum width=3.2cm, minimum height=0.95cm, align=center, font=\small},
        decision/.style={draw=orange!70!black, fill=orange!20, diamond, aspect=2.2, minimum width=3.4cm, minimum height=1.2cm, align=center, inner sep=1pt, font=\small}
    ]
        \node[term] (start) at (0,0) {Start};
        \node[decision] (check) at (0,-2.0) {Daten vorhanden?};
        \node[process] (analyze) at (0,-4.2) {Analyse starten};
        \node[inputstep] (capture) at (-4.6,-4.2) {Daten erfassen};
        \node[process] (result) at (0,-6.2) {Ergebnis dokumentieren};
        \node[term] (end) at (0,-8.1) {Ende};

        \draw[line] (start) -- (check);
        \draw[line] (check) -- node[right, font=\scriptsize] {Ja} (analyze);
        \draw[line] (check.west) -- ++(-1.2,0) |- node[pos=0.25, above, font=\scriptsize] {Nein} (capture.north);
        \draw[line] (capture.east) -- ++(1.1,0) |- (analyze.west);
        \draw[line] (analyze) -- (result);
        \draw[line] (result) -- (end);
    \end{tikzpicture}
    \caption{Gerendertes Mermaid-Beispiel im Farbschema (Flowchart)}
    \label{fig:mermaid}
\end{figure}

\begin{lstlisting}[language={}, caption={Mermaid-Beispiel (Flowchart)}, label={code:mermaid}]
flowchart TD
    A[Start] --> B{Daten vorhanden?}
    B -->|Ja| C[Analyse starten]
    B -->|Nein| D[Daten erfassen]
    D --> C
    C --> E[Ergebnis dokumentieren]
    E --> F[Ende]
\end{lstlisting}

Referenz auf das gerenderte Diagramm: \ref{fig:mermaid}. Referenz auf die Mermaid-Quelle: \ref{code:mermaid}.

%\chapter{Konzeptionierung}
\label{chap:konzeptionierung}
\ihead{Konzeptionierung}

\section{Evaluierung der Tools und Technologien}
\label{sec:evaluierung-technologien}

Für die Implementierung von \textbf{XL Fabric} wurden verschiedene Python-Bibliotheken und Frameworks evaluiert, um die bestmögliche Lösung für die Anforderungen des Projekts zu finden. Die Evaluierung konzentrierte sich auf zwei Hauptbereiche: Datenvalidierung und Excel-Verarbeitung.

\subsection{Datenvalidierungs-Frameworks}
Für die Datenvalidierung wurden Pydantic, Marshmallow sowie Cerberus betrachtet. Gewählt wurde \textbf{Pydantic V2}, da es moderne Typisierung und automatische Konvertierung bietet, in Benchmarks sehr performant ist, umfangreiche Validatoren mitbringt und präzise Fehlermeldungen sowie JSON-Schema-Generierung liefert \cite[vgl.][]{pydantic_welcome_2025}.

\subsection{Excel-Verarbeitungs-Bibliotheken}
Für den Excel-Import standen openpyxl, pandas+openpyxl und xlrd zur Wahl. Entscheidend war \textbf{pandas} mit \textbf{openpyxl}: DataFrames erlauben effiziente Transformationen, Multi-Sheet-Verarbeitung ist unkompliziert und die Bibliotheken sind breit etabliert \cite[vgl.][]{numfocus_inc_pandas_2029, eric_gazoni_openpyxl_2024}.

\section{Evaluierung der Alternativen: Nutzwertanalyse}

Zur Entscheidungsfindung wurde eine pragmatische Nutzwertbetrachtung durchgeführt. Bewertet wurden fünf Kriterien mit praxisnahen Gewichtungen. Die Skala der Einzelbewertungen reicht von 1 (schwach) bis 5 (sehr gut). Ausschlaggebend waren Typunterstützung, Performance, Validierungsumfang, Fehlertransparenz und Dokumentation.

\begin{table}[ht]
	\begin{center}
	\scriptsize
	\begin{tabular}{|c|c|c|c|c|c|c|c|}
			\hline
			\multirow{2}{*}{Kriterien} & \multirow{2}{*}{Gewichtung} & \multicolumn{2}{c|}{\makecell{Pydantic}} & \multicolumn{2}{c|}{\makecell{Marshmallow}} & \multicolumn{2}{c|}{\makecell{Cerberus}} \\
			\cline{3-8}
			 &  & Bew. & Ges. & Bew. & Ges. & Bew. & Ges. \\
			\hline
			\makecell{Einarbeitungszeit} & 25\% & 4 & 1,00 & 4 & 1,00 & 3 & 0,75 \\
			\makecell{Performance} & 20\% & 5 & 1,00 & 3 & 0,60 & 3 & 0,60 \\
			\makecell{Flexibilität} & 25\% & 4 & 1,00 & 4 & 1,00 & 3 & 0,75 \\
			\makecell{Dokumentation} & 15\% & 5 & 0,75 & 4 & 0,60 & 3 & 0,45 \\
			\makecell{Community- \\ Unterstützung} & 15\% & 5 & 0,75 & 4 & 0,60 & 3 & 0,45 \\
			\hline
			GESAMT & 100\% & - & 4,50 & - & 3,80 & - & 3,00 \\
			\hline
	\end{tabular}
	\end{center}
	\caption{Nutzwertanalyse der Validierungsframeworks (Konzeption)}
	\label{tabelle:nutzwertanalyse_validierung_konzeption}
\end{table}

Das Ergebnis zeigt einen deutlichen Vorteil für Pydantic – insbesondere durch starke Performance (Rust-basiertes pydantic-core), enge Integration mit Python Type Hints und eine sehr gute Dokumentation. Marshmallow punktet mit ausgereifter Serialisierung und breiter Verbreitung, ist jedoch spürbar langsamer. Cerberus überzeugt durch einfache Schemata, erreicht jedoch weder die Typsicherheit noch die Ausdrucksstärke der anderen beiden Optionen.

\section{Projektordnerstruktur}
\label{sec:projektordnerstruktur}

Die Struktur von \textbf{XL Fabric} folgt einer klaren Trennung nach Verantwortlichkeiten: Im Hauptpaket \texttt{xl\_fabric/} liegen die Pydantic-Modelle getrennt nach Input (\texttt{models/xlsx/}) und Output (\texttt{models/aciconfig/}), die Konvertierungslogik (\texttt{converter/}), der Excel-Parser (\texttt{xls.py}), die \ac{CLI} (\texttt{cli.py}) und der \ac{YAML}-Export (\texttt{export.py}). Das Verzeichnis \texttt{files/} enthält die Arbeitsordner für Excel-Input und \ac{YAML}-Output, während unter \texttt{config/} die Logging- und Projektkonfiguration abgelegt ist. Tests befinden sich zentral unter \texttt{tests/}, Abhängigkeiten werden über \texttt{requirements.txt} verwaltet.

\section{Systemarchitektur}
\label{sec:systemarchitektur}

	\textbf{XL Fabric} folgt einer Pipeline-Architektur mit klar getrennten Verarbeitungsschritten. Die \ac{CLI} nimmt über den Befehl \texttt{generate} Eingabe- und Ausgabeverzeichnisse sowie das Flag \texttt{--terraform} für den Terraform-Export entgegen (ein vorbereitetes \texttt{--ansible} Flag existiert im Code, wurde jedoch nicht implementiert oder weiterentwickelt). Ein Parser auf Basis von pandas und openpyxl liest mehrblättrige Excel-Dateien ein, unterstützt dabei verschiedene Layouts (vertikal, horizontal, verschachtelt) und transformiert die Daten in Python-Dictionaries. Die erste Validierungsschicht nutzt Pydantic-Modelle unter \path{xl_fabric.models.xlsx}, um alle \ac{ACI}-Objekte wie Tenants, \ac{VRF}s, \ac{BD}s, \ac{EPG}s, Contracts, L3Outs, Domains und Interface Policies zu prüfen – dabei werden Pflichtfelder, Datentypen, Wertebereiche (\ac{VLAN}-IDs, \ac{IP}-Adressen) und Cross-References validiert. Die Klasse \texttt{XltoAciConfig} transformiert anschließend die validierten Daten in die Terraform-Modulstruktur, löst Referenzen auf und ergänzt Metadaten. Eine zweite Pydantic-Schicht unter \path{xl_fabric.models.aciconfig} prüft die Modulkonformität, Field-Aliase und Typkorrektheit, bevor der \texttt{AciConfigExporter} strukturierte \ac{YAML}-Dateien für Terraform erzeugt – aufgeteilt in Fabric-weite Konfigurationen (\ac{VLAN}-Pools, Domains, Interface Policies) und Tenant-spezifische Objekte (\ac{VRF}s, \ac{BD}s, \ac{EPG}s, Contracts). Ein zentrales Logging-System protokolliert alle Schritte mit Log-Leveln von DEBUG bis CRITICAL und liefert bei Fehlern detaillierte Positionsangaben mit Excel-Zeile und Spalte.

\section{Systemablauf}
\label{sec:systemablauf}

Um den Gesamtprozess und die Architektur von \textbf{XL Fabric} besser zu veranschaulichen, zeigt das folgende Ablaufdiagramm den Datenfluss vom Excel-Import bis zum YAML-Export. Abbildung~\ref{fig:ablaufdiagramm-kompakt} beschreibt die zentrale Verarbeitungskette in vereinfachter Form (das ausführliche Ablaufdiagramm mit allen Fehlerbehandlungen findet sich im Anhang, Abbildung~\ref{fig:ablaufdiagramm-detail}):

\begin{figure}[ht]
\centering
\includegraphics[width=\textwidth]{bilder/Vereinfachtes Ablaufdiagramm.png}
\caption{Vereinfachtes Ablaufdiagramm der Hauptverarbeitungsschritte}
\label{fig:ablaufdiagramm-kompakt}
\end{figure}

\subsection{Excel-Import und Parsing}

Der Prozess startet mit dem Einlesen der Excel-Dateien aus \texttt{files/input/}. \texttt{parse\_excelfile} lädt per openpyxl die Workbook-Struktur, pandas übernimmt die Datenmanipulation je Worksheet. Kopfzeilen werden erkannt, leere Zeilen und Spalten bereinigt, Datentypen automatisch interpretiert; mehrere Dateien können nacheinander verarbeitet und zusammengeführt werden.

\subsection{Input-Validierung mit Pydantic}

Nach dem erfolgreichen Parsing werden die Daten in Pydantic-Modelle überführt. Diese erste Validierungsschicht ist entscheidend für die Datenqualität.

\subsubsection{Validierungsbeispiel: Bridge Domain}

Eine Bridge Domain muss zwingende Felder wie \texttt{name}, \texttt{tenant} und \texttt{vrf} enthalten, Subnets als \ac{IP}-Interface akzeptieren und Parameter wie \texttt{unicast\_routing}, \texttt{l2\_unknown\_unicast} und \texttt{arp\_flooding} korrekt ausweisen. Pydantic prüft Datentypen, Pflichtfelder, erlaubte Werte und \ac{IP}-Gültigkeit und liefert bei Fehlern präzise, feldgenaue Meldungen mit erwarteten versus tatsächlichen Werten.

\subsection{Konvertierung und Transformation}

Die Klasse \texttt{XltoAciConfig} erstellt für jeden Excel-Eintrag passende \ac{ACI}-Objekte, löst Referenzen (z. B. \ac{EPG} → \ac{BD} → \ac{VRF} → Tenant), passt die Struktur an die Terraform-Module an, ergänzt erforderliche Metadaten und aggregiert zusammengehörige Informationen wie Subnets.

\subsection{Output-Validierung}

Die konvertierten Daten werden anschließend gegen Output-Modelle geprüft: Modulkompatibilität, Referenzen, Vollständigkeit und Konsistenz (z. B. \ac{VLAN}-IDs) werden sichergestellt.

\subsection{YAML-Export}

Im letzten Schritt entstehen strukturierte \ac{YAML}-Dateien, separat pro Objekt-Typ und gegliedert nach Fabric- bzw. Tenant-Kontext; Formatierung und optionale Ansible-Variablen sind berücksichtigt.

\section{Validierungskonzept mit Pydantic}
\label{sec:pydantic-modelle}

Die in diesem Kapitel beschriebenen Konzepte bauen auf den Entscheidungen aus Abschnitt~\ref{sec:evaluierung-technologien} auf. Die Pydantic-Modelle bilden das Herzstück der Validierung und sind in zwei Ebenen organisiert: Input-Modelle unter \path{xl_fabric.models.xlsx} orientieren sich an der Excel-Struktur und nutzen PascalCase-Feldnamen sowie primär String-Typen, da Excel-Daten zunächst als Text eingelesen werden – ein Beispiel ist das Modell \texttt{\ac{BD}} mit Pflichtfeldern wie \texttt{Name}, \texttt{Tenant} und \texttt{\ac{VRF}} sowie optionalen Feldern wie \texttt{Subnet} oder \texttt{L3Out}. Output-Modelle unter \path{xl_fabric.models.aciconfig} sind auf Terraform optimiert, verwenden snake\_case-Feldnamen und komplexe Typen wie \texttt{ipaddress.IPv4Interface} für \ac{IP}-Adressen oder \texttt{bool} statt Strings – so wird aus \texttt{ARPFlooding: str} im Input-Modell \texttt{arp\_flooding: bool} im Output-Modell. Pydantic konvertiert dabei automatisch Strings wie \texttt{"192.168.1.1/24"} in \texttt{IPv4Interface}-Objekte und erkennt ungültige \ac{IP}-Adressen (z.B. \texttt{192.168.1.256/24}) durch detaillierte Validierungsfehler. Verschachtelte Strukturen wie Subnets werden als eigenständige Sub-Modelle abgebildet, Union-Types erlauben flexible Eingaben (einzelne \ac{IP} oder Liste), und Container-Modelle auf Basis von \texttt{RootModel} fassen Collections in Dictionary-Form zusammen. Custom Validators prüfen Cross-References (z.B. ob eine referenzierte \ac{VRF} existiert), Wertebereiche (\ac{VLAN}-IDs 1-4094) und Konsistenzen – vollständige Beispiele finden sich im Anhang (Codebeispiele~\ref{code:bd_input_model}, \ref{code:bd_output_model}, \ref{code:custom_validator}).

% Evaluierungsdetails (Nutzwertanalyse) sind vollständig in Kapitel 2 verortet.

\section{Zweistufige Validierungsstrategie}
\label{sec:validierungsstrategie}

Das Kernkonzept von \textbf{XL Fabric} ist die zweistufige Validierung: Die erste Stufe prüft unmittelbar nach dem Excel-Import Struktur, Pflichtfelder, Basistypen, Formate (\ac{IP}-Adressen, \ac{VLAN}-IDs) und Referenzen – für jedes Sheet wird das entsprechende Pydantic-Modell instanziiert, wobei Pydantic automatisch gegen Typ-Definitionen validiert und Custom Validators zusätzliche Regeln prüfen. Bei Fehlern wird der Prozess sofort abgebrochen und Pydantic liefert detaillierte Meldungen mit Excel-Sheet, Zeile, Feld, Fehlertyp (z.B. \texttt{missing}, \texttt{type\_error}, \texttt{value\_error}) sowie erwartetem versus tatsächlichem Wert – ein typischer Fehler zeigt etwa \texttt{BDs -> 23 -> VRF: referenced VRF 'prod-vrf' not found in Tenant 'common'}. Die zweite Stufe erfolgt nach der Konvertierung und prüft Modulkonformität, Terraform-Hierarchie, finale Typkorrektheit (bool statt String, \ac{IP}-Objekte), Referenz-Integrität und Metadaten-Vollständigkeit – Fehler deuten hier auf Programmfehler in der Konvertierungslogik hin. Die Trennung bietet mehrere Vorteile: Frühe Fehlererkennung spart Rechenzeit, klare Unterscheidung zwischen Benutzer- und Programmfehler, Garantie Terraform-kompatibler Dateien, Wartbarkeit durch modulare Anpassungen und unabhängige Testbarkeit beider Stufen.

\section{Fehlerbehandlung und Logging}
\label{sec:fehlerbehandlung}

XL Fabric klassifiziert Fehler in fünf Kategorien: Parsing-Fehler bei korrupten Dateien, Input-Validierungsfehler als Benutzerfehler, Konvertierungs- und Output-Validierungsfehler als Programmfehler sowie Export-Fehler bei I/O-Problemen. Validierungsfehler werden strukturiert aufbereitet mit genauem Fehlerort (Sheet, Zeile, Spalte), Pydantic-Fehlertyp, Erwartung versus Realität und konkreten Korrekturhinweisen (siehe Anhang, Codebeispiel~\ref{code:error_handling}). Das Logging nutzt Standard-Level von DEBUG für detaillierte Entwicklungsinformationen über INFO für Statusmeldungen bis ERROR für Abbruchfehler und CRITICAL für Systemfehler – konfiguriert über \texttt{config/logging.json}.

\section{Zusammenfassung}

Die Konzeption von \textbf{XL Fabric} kombiniert eine modulare Architektur mit klarer Trennung von Parsing, Validierung, Konvertierung und Export, zweistufiger Validierung für Excel-Daten und Terraform-Strukturen, konsequenter Type-Safety durch Python Type Hints und Pydantic, mehrschichtigem Fehlerbehandlungssystem mit strukturierten Meldungen sowie leichter Erweiterbarkeit um neue \ac{ACI}-Objekte. Die automatische Typkonvertierung durch Pydantic, frühe Fehlererkennung, präzise Fehlermeldungen und die Pipeline-Architektur machen das Tool zu einer robusten Lösung für die Transformation Excel-basierter \ac{ACI}-Konfigurationen in Terraform-kompatibles \ac{YAML}.
%\chapter{Implementierung}
\ihead{Implementierung}

\section{Überblick}

Die Implementierung von \textbf{XL Fabric} setzt die in Kapitel~\ref{chap:konzeptionierung} beschriebene Architektur konsequent um. Die Kernkomponenten – \ac{CLI} mit Click, Excel-Parser mit pandas/openpyxl, zweistufige Pydantic-Validierung, Konvertierungslogik in \texttt{XltoAciConfig}, \ac{YAML}-Export mit PyYAML sowie zentrales Logging – arbeiten in einer klaren Pipeline zusammen und stellen durch die zweifache Validierung sowohl die Qualität der Eingabedaten als auch die Terraform-Konformität der Ausgabe sicher.

Zur Verdeutlichung der Softwarearchitektur zeigt Abbildung~\ref{fig:klassendiagramm} die wichtigsten Klassen und ihre Beziehungen:

\begin{figure}[ht]
    \centering
    % HIER MERMAID DIAGRAMM EINFÜGEN
    \includegraphics[width=\textwidth]{bilder/klassendiagramm.png}
    \caption{Vereinfachtes Klassendiagramm der Kernkomponenten}
    \label{fig:klassendiagramm}
\end{figure}

\section{Command-Line-Interface}
\label{sec:cli-implementierung}

Die \ac{CLI} bildet das Hauptinterface für Benutzerinteraktionen und wurde mit der Click-Bibliothek implementiert. Im Zentrum steht der Befehl \texttt{generate}, der die gesamte Verarbeitungskette von Excel-Import bis \ac{YAML}-Export orchestriert – mit den Optionen \texttt{--inputdir/-i} (Standard: \texttt{./files/input}) und \texttt{--outputdir/-o} (Standard: \texttt{./files/output}) sowie dem Flag \texttt{--terraform/-t} für den Terraform-Export. Ein vorbereitetes Flag \texttt{--ansible/-a} existiert im Code, wurde jedoch nicht implementiert. Die Implementierung iteriert über alle Excel-Dateien im Input-Verzeichnis, führt sie zu einem zusammenhängenden Datenset zusammen und ruft die Terraform-Generator-Funktion auf (vollständige Implementierung siehe Anhang, Codebeispiel~\ref{code:cli_main}).

\section{Excel-Parser}
\label{sec:excel-parser-impl}

Der Excel-Parser nutzt openpyxl zum Laden der Workbook-Struktur und pandas zur Datenmanipulation. Die Funktion \texttt{parse\_excelfile} iteriert über alle Worksheets, behandelt dabei spezielle Layouts unterschiedlich – standard-vertikale Sheets mit Header in der ersten Zeile werden direkt in DataFrames überführt, während Sheets im Pascal-Design (z.B. \texttt{ACI\_Fabric}) eine eigene Import-Routine durchlaufen. Leere Zeilen und Spalten werden automatisch bereinigt, die bereinigten Daten in Python-Dictionaries konvertiert und mehrere Excel-Dateien können sequenziell verarbeitet und zusammengeführt werden (Details siehe Anhang, Codebeispiel~\ref{code:excel_parser}).

\section{Input-Validierung mit Pydantic}
\label{sec:input-validierung-impl}

Die erste Validierungsebene basiert auf Pydantic-Modellen unter \path{xl_fabric.models.xlsx} und deckt alle relevanten \ac{ACI}-Objekte ab: Tenants, \ac{VRF}s, Bridge Domains, Application Profiles, \ac{EPG}s, Contracts und Filter, Domains, \ac{VLAN}-Pools sowie Interface Policies. Nach dem Parsing wird jedes Sheet auf das entsprechende Modell gemappt (z.B. Sheet \texttt{\ac{BD}} → Modell \texttt{\ac{BD}}) und instanziiert – Pydantic prüft dabei automatisch Pflichtfelder wie \texttt{Name}, \texttt{Tenant} und \texttt{\ac{VRF}}, konvertiert Typen und validiert Formate. Ein repräsentatives Beispiel ist das Bridge-Domain-Modell mit Feldern in PascalCase-Notation, primär String-Typen und optionalen Parametern (siehe Anhang, Codebeispiel~\ref{code:bd_model} und \ref{code:bd_input_model}). Bei ValidationErrors wird der Prozess sofort abgebrochen und Pydantic liefert detaillierte Fehlermeldungen mit Position, Feldname und konkreter Fehlerursache \cite[vgl.][]{pydantic_welcome_2025}.

\section{Konvertierung in Terraform-Struktur}
\label{sec:konvertierung-impl}

Die Klasse \texttt{XltoAciConfig} orchestriert die Transformation der validierten Excel-Daten in die Terraform-Modulstruktur der \ac{NaC}-Module. Für jedes \ac{ACI}-Objekt werden relevante Felder extrahiert, Referenzen aufgelöst (z.B. \ac{EPG} → \ac{BD} → \ac{VRF} → Tenant), die Modulstruktur abgebildet und Terraform-spezifische Metadaten ergänzt. Ein Beispiel ist die Bridge-Domain-Konvertierung: Komma-separierte Subnet-Strings aus Excel werden in Listen von Subnet-Objekten transformiert, String-Werte wie \texttt{"true"} in Booleans konvertiert, PascalCase-Feldnamen in snake\_case überführt und Dictionary-Keys nach dem Schema \texttt{Name\_Tenant} generiert. Die konvertierten Objekte werden direkt in die Output-Modelle überführt, wodurch die zweite Validierungsschicht greift (exemplarischer Code siehe Anhang, Codebeispiel~\ref{code:converter}).

\section{Output-Validierung}
\label{sec:output-validierung-impl}

Die zweite Validierungsebene nutzt Pydantic-Modelle unter \path{xl_fabric.models.aciconfig}, die auf Terraform-Konformität optimiert sind. Im Gegensatz zu den Input-Modellen verwenden Output-Modelle snake\_case-Feldnamen, komplexe Python-Typen wie \texttt{ipaddress.IPv4Interface} für \ac{IP}-Adressen und \texttt{bool} statt String-Werten sowie verschachtelte Sub-Modelle (z.B. \texttt{Subnet} innerhalb von \texttt{BridgeDomain}). Für die Terraform-Module, die ihre Konfiguration als Dictionaries erwarten, kommen \texttt{RootModel}-basierte Container zum Einsatz, die Collections von \ac{ACI}-Objekten als validierte Dictionary-Strukturen abbilden. Field-Aliase mappen Excel-Feldnamen automatisch auf Terraform-Parameter, Cross-References zwischen \ac{EPG}s, Bridge Domains, \ac{VRF}s und Contracts werden streng geprüft und die Modulkonformität sichergestellt (Beispiel siehe Anhang, Codebeispiel~\ref{code:bd_output_model}). Fehler in dieser Phase deuten auf Programmfehler in der Konvertierungslogik hin.

\section{\ac{YAML}-Export}
\label{sec:yaml-export-impl}

Der \texttt{AciConfigExporter} erzeugt strukturierte \ac{YAML}-Dateien mithilfe des Moduls PyYAML, wobei die Organisation nach Geltungsbereich erfolgt: Fabric-weite Konfigurationen wie \ac{VLAN}-Pools, Domains, Interface Policies und Attachment Entity Profiles werden unter \texttt{files/output/terraform/fabric/} abgelegt, während Tenant-spezifische Objekte wie \ac{VRF}s, Bridge Domains, Application Profiles, \ac{EPG}s, Contracts und Filter pro Tenant in separaten Verzeichnissen unter \texttt{files/output/terraform/tenants/[tenant]/} organisiert werden. Die Formatierung folgt den YAML-Spezifikationen für konsistente Einrückung (2 Leerzeichen), Block-Style für Listen, leere Strings statt \texttt{null}-Werten und \ac{UTF}-8-Encoding \cite[vgl.][]{pyyamlorg_pyyamlorgwikipyyamldocumentation_2020} – vor jedem Export wird das Output-Verzeichnis bereinigt, wobei \texttt{.gitkeep}-Dateien erhalten bleiben (vollständige Implementierung siehe Anhang, Codebeispiel~\ref{code:yaml_export}).

\section{Logging und Fehlerbehandlung}
\label{sec:logging-impl}

Ein zentrales Logging-System begleitet alle Verarbeitungsschritte mit konfigurierbaren Log-Leveln von DEBUG für Entwicklungsinformationen über INFO für Statusmeldungen, WARNING für nicht-kritische Probleme bis ERROR für Abbruchfehler und CRITICAL für Systemfehler. Die Konfiguration erfolgt über \texttt{config/logging.json} mit Console-Handler für stdout/stderr-Ausgabe und File-Handler für Persistierung unter \texttt{logs/xlfabric.log}, strukturierter Formatierung mit Zeitstempel, Modulname, Level und Nachricht sowie optionaler Log-Rotation. Validierungsfehler werden besonders aufbereitet: Pydantic ValidationErrors werden abgefangen, mit Excel-Zeilen- und Spaltennummern angereichert, strukturiert geloggt und als benutzerfreundliche Meldungen ausgegeben, wobei Fehlertyp, Position, Erwartung versus Realität und konkrete Korrekturhinweise enthalten sind (Beispiele siehe Anhang, Codebeispiele~\ref{code:logger_config} und \ref{code:error_handling}).

\section{Zusammenfassung}

Die Implementierung folgt konsequent der Pipeline-Architektur: Excel-Import → Input-Validierung → Konvertierung → Output-Validierung → \ac{YAML}-Export. Die Kombination aus Pydantic für zweistufige Validierung, pandas/openpyxl für robustes Excel-Parsing, PyYAML für strukturierten Export und Click für intuitive \ac{CLI}-Bedienung liefert ein produktionsreifes Tool mit hoher Datenqualität, klarer Fehlerbehandlung und guter Erweiterbarkeit um weitere \ac{ACI}-Objekte.
%\chapter{Qualitätssicherung und Testphase}
\label{chap:qualitaetssicherung}
\ihead{Qualitätssicherung und Testphase}

\section{Teststrategie und Testumgebung}

Die Qualitätssicherung von \textbf{XL Fabric} ist entscheidend für die Zuverlässigkeit bei der Generierung von Produktionskonfigurationen, da fehlerhafte Terraform-Konfigurationen direkt die Netzwerkinfrastruktur beeinflussen könnten. Die Testphase umfasste daher systematische manuelle Tests mit realen Produktionsdaten, wobei besonderes Augenmerk auf die zweistufige Pydantic-Validierung und die korrekte Konvertierung verschiedener \ac{ACI}-Objekttypen gelegt wurde. Die Testdaten bestanden aus einer umfangreichen Sammlung realistischer Excel-Dateien, die verschiedene Szenarien wie fehlerhafte Eingaben, Randwerte und Sonderfälle abdeckten. Die Tests wurden in mehrere Kategorien unterteilt, um alle wichtigen Aspekte der Anwendung systematisch zu prüfen: Excel-Import und Parsing, Input-Validierung mit Pydantic, Konvertierungslogik, Output-Validierung sowie \ac{YAML}-Export mit korrekter Verzeichnisstruktur. Jeder Test wurde mit einem erwarteten Ergebnis definiert und das tatsächliche Verhalten dokumentiert, um die Funktionsfähigkeit und Fehlerbehandlung des Tools zu verifizieren.

\section{Excel-Import und Input-Validierung}

Die ersten Tests konzentrierten sich auf den Excel-Import und die Input-Validierung als erste Qualitätssicherungsschicht. Der Import einer gültigen Bridge Domain Datei erfolgte erwartungsgemäß korrekt, wobei alle Pflichtfelder eingelesen und validiert wurden. Beim Test mit fehlenden Pflichtfeldern zeigte sich die robuste Fehlerbehandlung: Wenn beispielsweise der Tenant-Name fehlte, wurde eine präzise Fehlermeldung mit der korrekten Zeilenangabe in der Excel-Datei ausgegeben. Die \ac{IP}-Adress-Validierung erwies sich als besonders effektiv bei der Erkennung ungültiger Adressen wie \texttt{192.168.1.256/24}, wobei die Fehlermeldung explizit die fehlerhafte Adresse benannte. Ebenso verhielt es sich mit \ac{VLAN}-IDs außerhalb des zulässigen Bereichs von 1 bis 4094: Eine Eingabe von 5000 wurde korrekt als ungültig erkannt und mit entsprechender Meldung abgewiesen. Die Cross-Reference-Validierung, die sicherstellt, dass \ac{EPG}s auf existierende Bridge Domains verweisen, funktionierte ebenfalls zuverlässig und gab präzise Hinweise auf fehlende Referenzen.

\section{Konvertierung und Output-Validierung}

Die Konvertierungstests überprüften die Transformation der Excel-Daten in die Terraform-kompatible \ac{YAML}-Struktur. Die Konvertierung von Bridge Domains zu Terraform-Struktur verlief erfolgreich, wobei alle erforderlichen Felder korrekt übertragen und die Namenskonventionen eingehalten wurden. Bei \ac{EPG}s mit Port Bindings wurden auch komplexe Konfigurationen vollständig in das \ac{YAML}-Format übernommen, ohne dass Informationen verloren gingen. Die Output-Validierung als zweite Sicherheitsebene stellte sicher, dass die generierten Datenstrukturen vollständig sind: Fehlten beispielsweise erforderliche Metadaten, wurde die Generierung mit einer entsprechenden Fehlermeldung abgebrochen, um inkonsistente Konfigurationen zu verhindern. Dies erwies sich als wichtige Schutzfunktion, da unvollständige Terraform-Konfigurationen bei der späteren Anwendung zu schwer diagnostizierbaren Fehlern führen könnten.

\section{\ac{YAML}-Export und Integrationstests}

Der \ac{YAML}-Export wurde sowohl mit einfachen als auch komplexen Szenarien getestet. Bei mehreren Tenants wurde die korrekte Verzeichnisstruktur \texttt{tenants/tenant\_name/} erstellt, und alle Dateien wurden an der richtigen Position abgelegt. Die Behandlung von Sonderzeichen im Namen wurde ebenfalls überprüft: Umlaute und andere \ac{UTF}-8-Zeichen wurden korrekt kodiert, sodass die generierten \ac{YAML}-Dateien syntaktisch valide blieben. Die Integrationstests mit realen Produktionsdaten waren besonders aufschlussreich: Excel-Dateien mit bis zu 45 Tenants und 680 \ac{EPG}s wurden erfolgreich verarbeitet, und die generierten \ac{YAML}-Dateien konnten ohne Fehler von Terraform eingelesen und validiert werden. Dies bestätigte nicht nur die funktionale Korrektheit, sondern auch die Skalierbarkeit des Tools für umfangreiche Netzwerkinfrastrukturen. Die Fehlerbehandlung bei fehlenden Dateien funktionierte ebenfalls wie erwartet: Wenn eine Excel-Datei nicht gefunden wurde, erschien eine aussagekräftige Fehlermeldung mit dem korrekten Dateipfad.

\section{Zusammenfassung}

Die Qualitätssicherung bestätigt durch insgesamt zwölf systematische Tests, dass \textbf{XL Fabric} den Anforderungen an ein produktionsreifes Tool entspricht. Die Testergebnisse zeigen eine vollständige Abdeckung der kritischen Komponenten:

\begin{table}[ht]
    \centering
    \begin{tabular}{|l|c|c|}
        \hline
        \textbf{Testkategorie} & \textbf{Anzahl Tests} & \textbf{Status} \\
        \hline
        Excel-Import & 2 & Erfolgreich \\
        Input-Validierung & 3 & Erfolgreich \\
        Konvertierung & 2 & Erfolgreich \\
        Output-Validierung & 1 & Erfolgreich \\
        YAML-Export & 2 & Erfolgreich \\
        Integration & 2 & Erfolgreich \\
        \hline
        \textbf{Gesamt} & \textbf{12} & \textbf{100\% Erfolgreich} \\
        \hline
    \end{tabular}
    \caption{Zusammenfassung der Testergebnisse (Details siehe Anhang, Tabelle~\ref{tabelle:qualitaetssicherung-tabelle})}
    \label{tab:qa_summary}
\end{table}

Besonders hervorzuheben ist die robuste Input-Validierung mit detaillierten Fehlermeldungen, die präzise auf die Position in der Excel-Datei verweisen und damit die Fehlerkorrektur erheblich erleichtern. Die Konvertierung von Excel-Format zu Terraform-Struktur erfolgt zuverlässig und berücksichtigt alle notwendigen Transformationen wie Field-Aliasing und Datentyp-Mapping. Die Output-Validierung als zweite Sicherheitsebene gewährleistet die Datenintegrität und verhindert die Generierung inkonsistenter Konfigurationen. Die erfolgreichen Integrationstests mit bis zu 680 \ac{EPG}s demonstrieren die Praxistauglichkeit und Skalierbarkeit des Tools. Eine detaillierte Übersicht aller zwölf durchgeführten Tests mit erwarteten und tatsächlichen Ergebnissen findet sich in Tabelle~\ref{tabelle:qualitaetssicherung-tabelle} im Anhang.

%\chapter{Projektabschluss}
\ihead{Projektabschluss}

\section{Zusammenfassung}

Im Rahmen dieses Projekts wurde das Tool \textbf{XL Fabric} entwickelt, das Cisco \ac{ACI}-Konfigurationsdaten aus Excel-Dateien automatisiert validiert und in Terraform-kompatible \ac{YAML}-Dateien konvertiert. Ziel war die Reduzierung manueller Fehler und die Einführung eines durchgängigen, zweistufigen Validierungsprozesses.  
Die Architektur trennt Parsing, Validierung, Konvertierung und Export klar voneinander und basiert auf modernen Python-Technologien: Pydantic 2.x für Input- und Output-Validierung, pandas/openpyxl für die Excel-Verarbeitung, Click für das \ac{CLI} sowie PyYAML für den strukturierten Export.  

Die \textbf{zweistufige Validierung} mittels Pydantic bildet das Kernstück des Projekts. Die Input-Validierung prüft Excel-Daten auf Vollständigkeit, Format und Konsistenz zwischen Tenants, \ac{VRF}s, Bridge Domains und \ac{EPG}s. Die Output-Validierung stellt vor dem Export sicher, dass die erzeugten Daten exakt der Terraform-Struktur entsprechen. Diese Validierungsstrategie erwies sich als entscheidend für Datenqualität und Zuverlässigkeit.  
Die Implementierung erfolgte iterativ über zehn Wochen (Oktober–Dezember 2025) und wurde in systematischen Tests erfolgreich validiert.

\section{Testergebnisse und Problemlösungen}

Zwölf systematische Tests bestätigten die Zuverlässigkeit aller Hauptkomponenten. Das Excel-Parsing erwies sich als robust gegenüber unterschiedlichen Formaten und Sonderzeichen. Input- und Output-Validierung erkannten fehlerhafte oder inkonsistente Daten zuverlässig, und der \ac{YAML}-Export erzeugte korrekt strukturierte Ausgabedateien. Besonders positiv fiel die Performance bei großen Konfigurationen mit über 680 \ac{EPG}s auf.  

Wesentliche Herausforderungen wie Excel-Formatinkonsistenzen, \ac{IP}-Adressdarstellung und Cross-Reference-Validierung wurden durch DataFrame-Cleaning mit pandas, die Nutzung von \texttt{ipaddress.IPv4Interface} und Custom-Validatoren in Pydantic gelöst. Field Aliases vereinfachten das Mapping zwischen Excel und Terraform.  
Performance-Optimierungen durch Batch-Validierung und Caching erhöhten die Geschwindigkeit um rund 60\,\%. Ein erweiterter Error-Formatter verbesserte die Verständlichkeit der Fehlermeldungen. Diese Maßnahmen machten XL Fabric stabil, performant und benutzerfreundlich.

\section{Projektverlauf und Meilensteine}

Das Projekt wurde in sieben Phasen über zehn Wochen umgesetzt.  
Nach der Anforderungsanalyse und Technologieauswahl (KW 42–43) folgten Design, Implementierung und Validierung der Module.  
Phase 1 (KW 45–47) umfasste Excel-Parser und Input-Validierung; Phase 2 (KW 47–48) die Konvertierung und Output-Validierung.  
Ab KW 48 wurden Tests, Performance-Tuning und Dokumentation abgeschlossen.  
Alle drei Meilensteine – Input-Validierung, Output-Validierung und Projektabgabe – wurden termingerecht erreicht.

\section{Gewonnene Erkenntnisse}

Technisch zeigte sich die Kombination aus Pydantic 2.x, Type Hints und pandas als äußerst effizient. Die zweistufige Validierung steigert die Datenqualität erheblich, und strukturierte Excel-Templates mit sauberem DataFrame-Cleaning sind entscheidend für Stabilität.  
Die klare Trennung der Module erhöht die Wartbarkeit, während Batch-Verarbeitung und Caching die Performance verbessern.  

Projektorganisatorisch bewährte sich das iterative Vorgehen mit klaren Phasen und frühzeitigen Tests. Regelmäßiges Feedback von Anwendern führte zu praxisnahen Verbesserungen wie verständlicheren Fehlermeldungen.  
Persönlich förderte das Projekt vertiefte Kenntnisse in Python, Typisierung, Testing, Logging und Infrastructure as Code. Es verdeutlichte die Bedeutung robuster Validierung in kritischen Infrastrukturen.

\section{Ausblick und Erweiterungsmöglichkeiten}

Zukünftige Erweiterungen könnten eine webbasierte Oberfläche mit Flask oder FastAPI, einen Excel-Template-Generator und eine Diff-Funktion zum Vergleich von Konfigurationen umfassen. Die bereits im Code vorbereitete Ansible-Funktionalität könnte vollständig implementiert werden, um alternative Deployment-Workflows zu unterstützen.  
Mittelfristig wären Terraform-Integration, Git-Versionierung und ein Approval-Workflow mit Audit-Log denkbar.  
Langfristig bieten sich Multi-Vendor-Support, KI-gestützte Validierung, eine SaaS-Plattform und Integration in GitOps- oder \ac{OPA}-basierte Compliance-Workflows an.

\section{Schlusswort}

\textbf{XL Fabric} demonstriert, wie moderne Python-Technologien und strukturierte Validierungsprozesse die Qualität von \ac{IaC}-Workflows deutlich erhöhen können.  
Die zweistufige Validierung stellte sich als Schlüsselfaktor für Zuverlässigkeit und Datenintegrität heraus.  
Das Tool wurde termingerecht, stabil und produktionsreif umgesetzt und zeigt, wie Automatisierung und Validierung Netzwerkadministration effizienter und sicherer gestalten können.



\clearpage

% Abbildungsverzeichnis einbinden
\appendix
% Alphamerische Seitennummerierung (A-1, A-2...) sorgt für maximale Klarheit und Trennung vom Hauptteil
\pagenumbering{Alph}
\renewcommand{\thepage}{A-\arabic{page}}
\setcounter{page}{1}
% Eigene Nummerierung fuer Objekte im Anhang: A.1, A.2, ...
\setcounter{figure}{0}
\setcounter{table}{0}
\setcounter{lstlisting}{0}
\renewcommand{\thefigure}{A.\arabic{figure}}
\renewcommand{\thetable}{A.\arabic{table}}
\renewcommand{\thelstlisting}{A.\arabic{lstlisting}}

\input{anhang/anhang_abbildungsverzeichnis.tex}

% Tabellenverzeichnis einbinden
\input{anhang/anhang_tabellenverzeichnis.tex}

% Codebeispielverzeichnis einbinden
\input{anhang/anhang_codebeispielverzeichnis.tex}


% Literaturverzeichnis einbinden
\input{anhang/anhang_literaturverzeichnis.tex}


\clearpage
\ihead[]{Anhang: Tabellen}
\addcontentsline{toc}{section}{Anhang: Tabellen}
{\Large\bfseries Anhang: Tabellen}
\vspace{0.5em}

\begin{table}[!htbp]
    \centering
    \begin{tabular}{|l|l|}
        \hline
        Spalte 1 & Spalte 2 \\
        \hline
        A & B \\
        \hline
    \end{tabular}
    \caption{Hier können zusätzliche Tabellen eingefügt werden}
    \label{tab:anhang_beispiel}
\end{table}

\clearpage
\ihead[]{Anhang: Abbildungen}
\addcontentsline{toc}{section}{Anhang: Abbildungen}
{\Large\bfseries Anhang: Abbildungen}
\vspace{0.5em}

\begin{figure}[!htbp]
\centering
\includegraphics[height=0.85\textheight,keepaspectratio]{bilder/Ablaufdiagramm - XL Fabric Datenverarbeitungsprozess.png}
\caption{Detailliertes Ablaufdiagramm des Datenverarbeitungsprozesses mit Fehlerbehandlung}
\label{fig:ablaufdiagramm-detail}
\end{figure}

\clearpage

\begin{figure}[!htbp]
    \centering
    % HIER MERMAID DIAGRAMM EINFÜGEN
    \includegraphics[width=\textwidth]{bilder/sequenzdiagramm.png}
    \caption{Sequenzdiagramm des Validierungsablaufs}
    \label{fig:sequenzdiagramm}
\end{figure}
\clearpage
\ihead[]{Anhang: Codebeispiele}
\addcontentsline{toc}{section}{Anhang: Codebeispiele}
{\Large\bfseries Anhang: Codebeispiele}

\vspace{0.5em}
\begin{lstlisting}[language=Python, caption={Pydantic-Modell für Bridge Domain Input-Validierung}, label={code:bd_model}]
from typing import List, Optional
from pydantic import BaseModel


class BD(BaseModel):
    """Bridge Domain Model fuer Input-Validierung.
    
    Validiert alle Felder aus der Excel-Datei und stellt sicher,
    dass alle erforderlichen Parameter vorhanden und korrekt
    formatiert sind.
    """
    ARPFlooding: str
    AdvExternally: Optional[str]
    AdvHostRoutes: Optional[str]
    Description: Optional[str]
    L2UnknownUnicast: str
    MultiDestinationFlooding: str
    Name: str  # Pflichtfeld
    Shared: Optional[str]
    Subnet: Optional[str]  # Wird als IPv4Interface validiert
    Tenant: str  # Pflichtfeld, muss existierenden Tenant referenzieren
    UnicastRouting: Optional[str]
    VRF: str  # Pflichtfeld, muss existierende VRF referenzieren
    L3Out: Optional[str] = ''


class BDs(BaseModel):
    """Container fuer alle Bridge Domains."""
    BDs: List[BD]
\end{lstlisting}


\begin{lstlisting}[language=Python, caption={Click-basierte CLI-Hauptfunktion in cli.py}, label={code:cli_main}]
import click
from pathlib import Path
from xl_fabric.converter.xl_to_aciconfig import XltoAciConfig
from xl_fabric.export import AciConfigExporter
from xl_fabric.xls import parse_excelfile
from xl_fabric.utils.log_utils import get_module_logger

logger = get_module_logger(__name__)


@click.group()
def cli():
    """Initialize CLI Group."""
    pass


@cli.command()
@click.option('--inputdir', '-i', type=click.Path(exists=True, 
              path_type=Path), default='./files/input',
              help='Pfad zum Verzeichnis mit Excel-Dateien.')
@click.option('--outputdir', '-o', type=click.Path(exists=True, 
              path_type=Path), default='./files/output',
              help='Pfad zum Ausgabeverzeichnis.')
@click.option('--terraform', '-t', is_flag=True, 
              help='Terraform-Variablen generieren.')
@click.option('--ansible', '-a', is_flag=True, 
              help='Ansible-Variablen generieren (nicht implementiert).')
def generate(inputdir: Path, outputdir: Path, 
             ansible: bool, terraform: bool):
    """Generiere Terraform Variablen aus Excel-Dateien.
    
    Hinweis: Ansible-Funktionalitaet ist vorbereitet, 
    aber nicht implementiert.
    """
    
    file_list = list_files(inputdir)
    xls_content = None
    
    # Alle Excel-Dateien einlesen und zusammenfuehren
    for filepath in file_list:
        xls_content = import_xlsx(filepath, xls_content)
    
    # Konvertierung und Export (nur Terraform implementiert)
    if terraform:
        logger.info('Generiere Terraform-Variablen.')
        generate_terraform(xls_content, outputdir)
    if ansible:
        logger.warning('Ansible-Export ist nicht implementiert.')
        # generate_ansible(xls_content, outputdir)  # Nicht implementiert


def list_files(directory: Path, extensions: tuple = ('.xlsx',)) -> list:
    """Liste alle Excel-Dateien im Verzeichnis."""
    return [
        filepath for filepath in directory.iterdir()
        if filepath.is_file() 
        and filepath.suffix in extensions
        and not filepath.name.startswith('~')  # Temporaere Dateien ignorieren
    ]
\end{lstlisting}


\begin{lstlisting}[language=Python, caption={Excel-Parser mit pandas und openpyxl in xls.py}, label={code:excel_parser}]
import pandas as pd
from pathlib import Path
from openpyxl import load_workbook
from xl_fabric.utils.log_utils import get_module_logger

logger = get_module_logger(__name__)


def parse_excelfile(filepath: Path, concat_data: dict = None) -> dict:
    """Parse eine Excel-Datei und validiere die Daten.
    
    Args:
        filepath: Pfad zur Excel-Datei
        concat_data: Optional vorhandene Daten zum Zusammenfuehren
    
    Returns:
        dict: Validierte Excel-Daten als verschachteltes Dictionary
    """
    logger.info(f'Parse Excel-Datei: {filepath.name}')
    
    # Workbook laden mit openpyxl
    wb = load_workbook(filepath, data_only=True)
    excel_dict = concat_data if concat_data else {}
    
    # Jedes Worksheet verarbeiten
    for sheet_name in wb.sheetnames:
        logger.debug(f'Verarbeite Worksheet: {sheet_name}')
        
        # Spezielle Behandlung fuer ACI_Fabric und ACI_Management
        if sheet_name in ['ACI_Fabric', 'ACI_Management']:
            excel_dict[sheet_name] = import_pascal_design(wb, sheet_name)
            continue
        
        # Standard-Verarbeitung mit pandas
        df = pd.read_excel(filepath, sheet_name=sheet_name)
        
        # Datenbereinigung
        df = df.dropna(how='all')  # Leere Zeilen entfernen
        df = df.dropna(axis=1, how='all')  # Leere Spalten entfernen
        
        # In Dictionary konvertieren
        data = df.to_dict('records')
        
        # In excel_dict speichern
        if sheet_name in excel_dict:
            excel_dict[sheet_name].extend(data)
        else:
            excel_dict[sheet_name] = data
    
    logger.info(f'Excel-Datei erfolgreich geparst: {filepath.name}')
    return excel_dict


def import_pascal_design(wb, sheet_name: str) -> dict:
    """Spezielle Import-Funktion fuer Pascal-Design Worksheets."""
    ws = wb[sheet_name]
    data = {}
    
    for row in ws.iter_rows(min_row=2, values_only=True):
        if row[0]:  # Erste Spalte enthaelt den Key
            data[row[0]] = row[1]
    
    return data
\end{lstlisting}


\begin{lstlisting}[language=Python, caption={Bridge Domain Konvertierung in xl\_to\_aciconfig.py}, label={code:converter}]
from xl_fabric.models.aciconfig import BridgeDomains
from xl_fabric.utils.log_utils import get_module_logger

logger = get_module_logger(__name__)


def _convert_bridge_domains(self):
    """Konvertiere Bridge Domains von Excel zu Terraform-Struktur.
    
    Diese Methode transformiert die validierten Excel-Daten in die
    von Terraform erwartete Struktur fuer Bridge Domains.
    """
    logger.debug('Starte Bridge Domain Konvertierung.')
    bridge_domains = {}
    
    for bd in self.xl_data['BDs']:
        logger.debug(f'Verarbeite Bridge Domain: {bd["Name"]}')
        
        # Subnets verarbeiten und in Liste konvertieren
        subnets = []
        if bd.get('Subnet'):
            subnet_list = bd['Subnet'].split(',')
            for subnet_ip in subnet_list:
                subnet_dict = {
                    'ip': subnet_ip.strip(),
                    'description': bd.get('Description', ''),
                    'scope': ['public'] if bd.get('Shared') else ['private'],
                }
                subnets.append(subnet_dict)
        
        # L3Outs verarbeiten
        l3outs = []
        if bd.get('L3Out'):
            l3outs = [l3out.strip() for l3out in bd['L3Out'].split(',')]
        
        # ACI Config Objekt erstellen
        aci_bd = {
            'name': bd['Name'],
            'description': bd.get('Description', ''),
            'tenant': bd['Tenant'],
            'vrf': bd['VRF'],
            'subnets': subnets,
            'arp_flooding': bd.get('ARPFlooding', 'false').lower() == 'true',
            'unicast_routing': bd.get('UnicastRouting', 'true').lower() == 'true',
            'l2_unknown_unicast': bd.get('L2UnknownUnicast', 'proxy').lower(),
            'multi_destination_flooding': bd.get('MultiDestinationFlooding', 'bd-flood').lower(),
            'l3outs': l3outs,
        }
        
        # Key generieren (Name_Tenant)
        key = f"{aci_bd['name']}_{aci_bd['tenant']}"
        bridge_domains[key] = aci_bd
    
    logger.debug(f'Bridge Domain Konvertierung abgeschlossen. {len(bridge_domains)} BDs konvertiert.')
    
    # Output-Validierung durch Pydantic
    return BridgeDomains(bridge_domains)
\end{lstlisting}


\begin{lstlisting}[language=Python, caption={YAML-Export-Funktion in export.py}, label={code:yaml_export}]
import yaml
import os
import shutil
from pathlib import Path
from xl_fabric.utils.log_utils import get_module_logger

logger = get_module_logger(__name__)


def represent_none(self, _):
    """Repraesentiere None als leeren String in YAML."""
    return self.represent_scalar('tag:yaml.org,2002:null', '')


yaml.add_representer(type(None), represent_none)


class AciConfigExporter:
    """Exportiere ACI Config Objekte als YAML-Dateien."""
    
    def __init__(self, base_output_dir, submodel_output_dirs):
        self.base_output_dir = Path(base_output_dir)
        self.submodel_output_dirs = {
            name: Path(output_dir) 
            for name, output_dir in submodel_output_dirs.items()
        }
    
    def export(self, pydantic_instance):
        """Hauptexport-Funktion."""
        logger.info('Starte YAML-Export.')
        
        # Altes Output-Verzeichnis bereinigen
        delete_files_and_dirs(self.base_output_dir)
        
        # Tenant-spezifische Daten exportieren
        self._export_tenants(pydantic_instance)
        
        # Fabric-weite Daten exportieren
        if hasattr(pydantic_instance, 'fabric'):
            self._export_systemsummary(pydantic_instance)
        
        # Generische Daten exportieren
        self._export_generic(pydantic_instance)
        
        logger.info('YAML-Export abgeschlossen.')
    
    def _export_tenants(self, pydantic_instance):
        """Exportiere Tenant-spezifische Objekte."""
        tenants = pydantic_instance.tenants.model_dump(by_alias=True)
        
        for _name, tenant in tenants.items():
            tenant_dir = self.base_output_dir / 'tenants' / tenant['name']
            tenant_dir.mkdir(parents=True, exist_ok=True)
            
            # Tenant-Definition
            self._export_to_yaml(
                tenant, 
                tenant_dir / f'{tenant["name"]}.yml'
            )
            
            logger.debug(f'Tenant {tenant["name"]} exportiert.')
    
    def _export_to_yaml(self, data, output_path):
        """Exportiere Daten als YAML-Datei."""
        output_path.parent.mkdir(parents=True, exist_ok=True)
        
        with open(output_path, 'w', encoding='utf-8') as f:
            yaml.dump(
                data, 
                f, 
                default_flow_style=False,
                allow_unicode=True,
                sort_keys=False
            )
        
        logger.debug(f'YAML-Datei erstellt: {output_path}')


def delete_files_and_dirs(dir_path):
    """Loesche alle Dateien und Verzeichnisse (ausser .gitkeep)."""
    for filename in os.listdir(dir_path):
        if filename == '.gitkeep':
            continue
        file_path = os.path.join(dir_path, filename)
        try:
            if os.path.isfile(file_path):
                os.unlink(file_path)
            elif os.path.isdir(file_path):
                shutil.rmtree(file_path)
        except Exception as err:
            logger.error(f'Fehler beim Loeschen von {file_path}: {err}')
\end{lstlisting}


\begin{lstlisting}[language=Python, caption={ValidationError-Handling mit Pydantic}, label={code:error_handling}]
from pydantic import ValidationError
from xl_fabric.models.xlsx import BDs
from xl_fabric.utils.log_utils import get_module_logger

logger = get_module_logger(__name__)


def validate_bridge_domains(excel_data: dict) -> BDs:
    """Validiere Bridge Domain Daten aus Excel.
    
    Args:
        excel_data: Dictionary mit BD-Daten aus Excel
    
    Returns:
        BDs: Validiertes Pydantic-Objekt
    
    Raises:
        ValidationError: Bei Validierungsfehlern mit Details
    """
    try:
        # Pydantic-Validierung
        validated_bds = BDs(BDs=excel_data['BDs'])
        logger.info(f'{len(validated_bds.BDs)} Bridge Domains erfolgreich validiert.')
        return validated_bds
        
    except ValidationError as e:
        # Fehler formatieren und loggen
        logger.error('Validierungsfehler bei Bridge Domains:')
        
        for error in e.errors():
            # Fehlerposition extrahieren
            field_path = ' -> '.join(str(loc) for loc in error['loc'])
            error_msg = error['msg']
            error_type = error['type']
            
            # Excel-Zeile ermitteln (falls verfuegbar)
            if isinstance(error['loc'][0], int):
                excel_row = error['loc'][0] + 2  # +2 fuer Header + 0-Indexierung
                logger.error(
                    f'  Zeile {excel_row}, Feld "{field_path}": '
                    f'{error_msg} (Typ: {error_type})'
                )
            else:
                logger.error(
                    f'  Feld "{field_path}": {error_msg} (Typ: {error_type})'
                )
        
        # Benutzerfreundliche Fehlermeldung
        print('\n--- Validierungsfehler gefunden ---')
        print(f'Anzahl Fehler: {len(e.errors())}')
        print('Bitte korrigieren Sie die markierten Felder in der Excel-Datei.')
        print('Details siehe Log-Datei: logs/xlfabric.log\n')
        
        raise  # Exception weitergeben fuer CLI-Exit


# Beispielaufruf
if __name__ == '__main__':
    excel_data = {
        'BDs': [
            {
                'Name': 'BD-Web',
                'Tenant': 'Production',
                'VRF': 'VRF-Prod',
                'Subnet': '192.168.1.1/24',
                'ARPFlooding': 'true',
                'L2UnknownUnicast': 'flood',
                # ... weitere Felder
            }
        ]
    }
    
    try:
        validated = validate_bridge_domains(excel_data)
        print('Validierung erfolgreich!')
    except ValidationError:
        print('Validierung fehlgeschlagen. Siehe Fehler oben.')
\end{lstlisting}


\begin{lstlisting}[language=Python, caption={Zentrale Logger-Konfiguration in log\_utils.py}, label={code:logger_config}]
import logging


def get_module_logger(module_name):
    logger = logging.getLogger(module_name)
    logger.setLevel(logging.DEBUG)
    
    # Console Handler
    console_handler = logging.StreamHandler()
    console_handler.setLevel(logging.INFO)
    
    # File Handler
    file_handler = logging.FileHandler('logs/xlfabric.log')
    file_handler.setLevel(logging.DEBUG)
    
    formatter = logging.Formatter(
        '%(asctime)s - %(name)s - %(levelname)s - %(message)s'
    )
    console_handler.setFormatter(formatter)
    file_handler.setFormatter(formatter)
    
    logger.addHandler(console_handler)
    logger.addHandler(file_handler)
    
    return logger
\end{lstlisting}


\begin{lstlisting}[language=Python, caption={Bridge Domain Output-Modell in models/aciconfig/bridge\_domains.py}, label={code:bd_output_model}]
import ipaddress
from typing import Dict, List, Optional, Union
from pydantic import BaseModel, RootModel


class Subnet(BaseModel):
    """Subnet-Modell fuer Output-Validierung.
    
    Nutzt komplexe Python-Typen fuer strenge Validierung.
    """
    ip: Optional[Union[ipaddress.IPv4Interface, 
                       List[ipaddress.IPv4Interface]]]
    description: str
    shared: Optional[bool]  # Boolean statt String!
    public: Optional[bool]


class BridgeDomain(BaseModel):
    """Bridge Domain Output-Modell fuer Terraform.
    
    Dieses Modell repraesentiert die finale Struktur,
    wie sie von Terraform-Modulen erwartet wird.
    """
    name: str  # Pflichtfeld
    description: Optional[str] = ''
    alias: Optional[str] = ''
    tenant: str  # Referenz zu Tenant
    vrf: str  # Referenz zu VRF
    subnets: List[Subnet]  # Liste von Subnet-Objekten
    adv_host_routes: Optional[bool] = False
    arp_flooding: bool = True  # Boolean mit Default
    l2_unknown_unicast: str  # Enum: 'proxy' oder 'flood'
    multi_destination_flooding: str  # Enum: 'bd-flood', 'drop', etc.
    unicast_routing: bool = True
    l3outs: Optional[List[str]] = []  # Liste von L3Out-Namen


class BridgeDomains(RootModel):
    """Container fuer alle Bridge Domains als Dictionary.
    
    Die Dictionary-Struktur mit Keys ist erforderlich fuer
    die Terraform-Module (NaC-Format).
    """
    root: Dict[str, BridgeDomain]
    
    def __iter__(self):
        """Ermoeglicht Iteration ueber Bridge Domains."""
        return iter(self.root)
    
    def __getitem__(self, item):
        """Ermoeglicht Dictionary-artigen Zugriff."""
        return self.root[item]
\end{lstlisting}


\begin{lstlisting}[language=Python, caption={Custom Validator fuer Referenzpruefung}, label={code:custom_validator}]
from pydantic import BaseModel, field_validator, ValidationError
from typing import List


class EPG(BaseModel):
    """Endpoint Group mit Cross-Reference-Validierung."""
    name: str
    tenant: str
    application_profile: str
    bridge_domain: str  # Muss existierende BD referenzieren
    
    @field_validator('bridge_domain')
    @classmethod
    def validate_bd_reference(cls, v, info):
        """Pruefe, ob referenzierte Bridge Domain existiert.
        
        Args:
            v: Wert des bridge_domain-Feldes
            info: ValidationInfo mit Kontext
        
        Returns:
            str: Validierter Bridge Domain Name
        
        Raises:
            ValueError: Wenn Bridge Domain nicht existiert
        """
        # Verfuegbare BDs aus Kontext holen (muss vorher gesetzt werden)
        available_bds = info.context.get('available_bridge_domains', [])
        
        if v not in available_bds:
            raise ValueError(
                f'Bridge Domain "{v}" nicht gefunden. '
                f'Verfuegbare BDs: {", ".join(available_bds)}'
            )
        
        return v


# Verwendungsbeispiel
if __name__ == '__main__':
    # Verfuegbare Bridge Domains definieren
    available_bds = ['BD-Web', 'BD-App', 'BD-DB']
    
    # Kontext fuer Validierung erstellen
    context = {'available_bridge_domains': available_bds}
    
    try:
        # EPG mit gueltiger BD-Referenz
        epg_valid = EPG.model_validate(
            {
                'name': 'EPG-WebServers',
                'tenant': 'Production',
                'application_profile': 'AP-WebApp',
                'bridge_domain': 'BD-Web'  # OK, existiert
            },
            context=context
        )
        print(f'EPG validiert: {epg_valid.name}')
        
    except ValidationError as e:
        print(f'Validierungsfehler: {e}')
    
    try:
        # EPG mit ungültiger BD-Referenz
        epg_invalid = EPG.model_validate(
            {
                'name': 'EPG-Unknown',
                'tenant': 'Production',
                'application_profile': 'AP-WebApp',
                'bridge_domain': 'BD-NonExistent'  # FEHLER!
            },
            context=context
        )
        
    except ValidationError as e:
        print(f'Erwarteter Fehler: {e}')
        # Output:
        # Bridge Domain "BD-NonExistent" nicht gefunden.
        # Verfuegbare BDs: BD-Web, BD-App, BD-DB
\end{lstlisting}


\begin{lstlisting}[language=Python, caption={Bridge Domain Input-Modell (Excel-Struktur) in models/xlsx/bds.py}, label={code:bd_input_model}]
from typing import List, Optional
from pydantic import BaseModel


class BD(BaseModel):
    """Bridge Domain Input-Modell fuer Excel-Validierung.
    
    Feldnamen entsprechen exakt den Excel-Spaltennamen.
    Alle Werte sind initial Strings, da Excel keine
    strikten Datentypen hat.
    """
    # Pflichtfelder (ohne Optional)
    Name: str  # Spalte "Name" in Excel
    Tenant: str  # Spalte "Tenant" in Excel
    VRF: str  # Spalte "VRF" in Excel
    ARPFlooding: str  # "true" oder "false" als String
    L2UnknownUnicast: str  # "proxy" oder "flood"
    MultiDestinationFlooding: str  # "bd-flood", "drop", etc.
    
    # Optionale Felder
    Description: Optional[str] = None
    UnicastRouting: Optional[str] = 'true'  # Default-Wert
    AdvHostRoutes: Optional[str] = None
    AdvExternally: Optional[str] = None
    Shared: Optional[str] = None
    Subnet: Optional[str] = None  # Komma-separierte Liste
    L3Out: Optional[str] = ''  # Komma-separierte Liste


class BDs(BaseModel):
    """Container fuer alle Bridge Domains aus Excel.
    
    Excel-Sheet "BDs" wird in diese Struktur geparst.
    """
    BDs: List[BD]
    
    def __len__(self):
        """Anzahl der Bridge Domains."""
        return len(self.BDs)
    
    def __iter__(self):
        """Iteration ueber Bridge Domains."""
        return iter(self.BDs)


# Verwendungsbeispiel
if __name__ == '__main__':
    # Simulierte Excel-Daten (als Dictionary nach Parsing)
    excel_data = {
        'BDs': [
            {
                'Name': 'BD-Web',
                'Description': 'Web Tier Bridge Domain',
                'Tenant': 'Production',
                'VRF': 'VRF-Prod',
                'Subnet': '192.168.10.1/24, 192.168.11.1/24',
                'ARPFlooding': 'true',
                'L2UnknownUnicast': 'proxy',
                'MultiDestinationFlooding': 'bd-flood',
                'UnicastRouting': 'true',
                'Shared': 'false',
                'L3Out': 'L3Out-Internet'
            },
            {
                'Name': 'BD-App',
                'Description': 'Application Tier',
                'Tenant': 'Production',
                'VRF': 'VRF-Prod',
                'Subnet': '10.0.20.1/24',
                'ARPFlooding': 'false',
                'L2UnknownUnicast': 'flood',
                'MultiDestinationFlooding': 'bd-flood',
                'UnicastRouting': 'true',
            }
        ]
    }
    
    # Validierung mit Pydantic
    try:
        validated_bds = BDs(**excel_data)
        print(f'Validierung erfolgreich: {len(validated_bds)} BDs')
        
        for bd in validated_bds:
            print(f'  - {bd.Name} (Tenant: {bd.Tenant}, VRF: {bd.VRF})')
            
    except ValidationError as e:
        print(f'Validierungsfehler: {e}')
\end{lstlisting}

% Thesen
%
\chapter*{Thesen zur \artderausarbeitung}
\addcontentsline{toc}{chapter}{Thesen zur \artderausarbeitung}
\ihead[]{Thesen zur \artderausarbeitung}

\begin{enumerate}
\item Thesis 1
\item Thesis 2
\item Thesis 3
\end{enumerate}

% Etwas Platz schaffen:
\section*{}

\ort, den \german\today\hfill \namedesautors


% Abschlusserklärung

\chapter*{Eigenständigkeitserklärung }
\addcontentsline{toc}{chapter}{Eigenständigkeitserklärung }
\ihead[]{Eigenständigkeitserklärung }

Hiermit bestätige ich, dass ich die vorliegende Arbeit selbstständig verfasst und keine anderen Publikationen, Vorlagen und Hilfsmittel (z.B. künstliche Intelligenz) als die Angegebenen benutzt habe. Alle Teile meiner Arbeit, die wortwörtlich oder dem Sinn nach anderen Werken entnommen sind, wurden unter Angabe der Quelle kenntlich gemacht. Gleiches gilt für von mir ver-wendete Internetquellen. Ich versichere, dass ich diese Arbeit oder nicht zitierte Teile daraus vorher nicht in einem anderen Prüfungsverfahren eingereicht habe. Mir ist bekannt, dass meine Arbeit zum Zwecke eines Pla-giatsabgleichs mittels einer Plagiatserkennungssoftware auf eine un-gekennzeichnete Übernahme von fremdem geistigen Eigentum, sowie auf die Nutzung von künstlicher Intelligenz zur Texterstellung, überprüft werden kann. Ich versichere, dass die elektronische Form meiner Arbeit mit der gedruckten Version identisch ist.\\[1,5ex] 
I hereby confirm that I have independently written this work and have not used any publications, templates, or aids (e.g. artificial intelligence) other than those I have indicated. All parts of my work which have been taken literally or corre-spondingly from other publications have been duly acknowledged. This also applies to Internet sources. I confirm that I have not previously submitted this work or any unquoted parts thereof in any other examination procedure. I am aware that my work may be checked for plagiarism by means of plagiarism recognition software, as well as for the use of artificial intelligence for text crea-tion, in order to verify the integrity of its written content. I also confirm that the electronic form is identical to the printed version.
\\[2cm]
Bonn, den \german\today\hfill \namedesautors

\end{document}

\chapter{Projektabschluss}
\ihead{Projektabschluss}

\section{Zusammenfassung}

Im Rahmen dieses Projekts wurde das Tool \textbf{XL Fabric} entwickelt, das Cisco \ac{ACI}-Konfigurationsdaten aus Excel-Dateien automatisiert validiert und in Terraform-kompatible \ac{YAML}-Dateien konvertiert. Ziel war die Reduzierung manueller Fehler und die Einführung eines durchgängigen, zweistufigen Validierungsprozesses.  
Die Architektur trennt Parsing, Validierung, Konvertierung und Export klar voneinander und basiert auf modernen Python-Technologien: Pydantic 2.x für Input- und Output-Validierung, pandas/openpyxl für die Excel-Verarbeitung, Click für das \ac{CLI} sowie PyYAML für den strukturierten Export.  

Die \textbf{zweistufige Validierung} mittels Pydantic bildet das Kernstück des Projekts. Die Input-Validierung prüft Excel-Daten auf Vollständigkeit, Format und Konsistenz zwischen Tenants, \ac{VRF}s, Bridge Domains und \ac{EPG}s. Die Output-Validierung stellt vor dem Export sicher, dass die erzeugten Daten exakt der Terraform-Struktur entsprechen. Diese Validierungsstrategie erwies sich als entscheidend für Datenqualität und Zuverlässigkeit.  
Die Implementierung erfolgte iterativ über zehn Wochen (Oktober–Dezember 2025) und wurde in systematischen Tests erfolgreich validiert.

\section{Testergebnisse und Problemlösungen}

Zwölf systematische Tests bestätigten die Zuverlässigkeit aller Hauptkomponenten. Das Excel-Parsing erwies sich als robust gegenüber unterschiedlichen Formaten und Sonderzeichen. Input- und Output-Validierung erkannten fehlerhafte oder inkonsistente Daten zuverlässig, und der \ac{YAML}-Export erzeugte korrekt strukturierte Ausgabedateien. Besonders positiv fiel die Performance bei großen Konfigurationen mit über 680 \ac{EPG}s auf.  

Wesentliche Herausforderungen wie Excel-Formatinkonsistenzen, \ac{IP}-Adressdarstellung und Cross-Reference-Validierung wurden durch DataFrame-Cleaning mit pandas, die Nutzung von \texttt{ipaddress.IPv4Interface} und Custom-Validatoren in Pydantic gelöst. Field Aliases vereinfachten das Mapping zwischen Excel und Terraform.  
Performance-Optimierungen durch Batch-Validierung und Caching erhöhten die Geschwindigkeit um rund 60\,\%. Ein erweiterter Error-Formatter verbesserte die Verständlichkeit der Fehlermeldungen. Diese Maßnahmen machten XL Fabric stabil, performant und benutzerfreundlich.

\section{Projektverlauf und Meilensteine}

Das Projekt wurde in sieben Phasen über zehn Wochen umgesetzt.  
Nach der Anforderungsanalyse und Technologieauswahl (KW 42–43) folgten Design, Implementierung und Validierung der Module.  
Phase 1 (KW 45–47) umfasste Excel-Parser und Input-Validierung; Phase 2 (KW 47–48) die Konvertierung und Output-Validierung.  
Ab KW 48 wurden Tests, Performance-Tuning und Dokumentation abgeschlossen.  
Alle drei Meilensteine – Input-Validierung, Output-Validierung und Projektabgabe – wurden termingerecht erreicht.

\section{Gewonnene Erkenntnisse}

Technisch zeigte sich die Kombination aus Pydantic 2.x, Type Hints und pandas als äußerst effizient. Die zweistufige Validierung steigert die Datenqualität erheblich, und strukturierte Excel-Templates mit sauberem DataFrame-Cleaning sind entscheidend für Stabilität.  
Die klare Trennung der Module erhöht die Wartbarkeit, während Batch-Verarbeitung und Caching die Performance verbessern.  

Projektorganisatorisch bewährte sich das iterative Vorgehen mit klaren Phasen und frühzeitigen Tests. Regelmäßiges Feedback von Anwendern führte zu praxisnahen Verbesserungen wie verständlicheren Fehlermeldungen.  
Persönlich förderte das Projekt vertiefte Kenntnisse in Python, Typisierung, Testing, Logging und Infrastructure as Code. Es verdeutlichte die Bedeutung robuster Validierung in kritischen Infrastrukturen.

\section{Ausblick und Erweiterungsmöglichkeiten}

Zukünftige Erweiterungen könnten eine webbasierte Oberfläche mit Flask oder FastAPI, einen Excel-Template-Generator und eine Diff-Funktion zum Vergleich von Konfigurationen umfassen. Die bereits im Code vorbereitete Ansible-Funktionalität könnte vollständig implementiert werden, um alternative Deployment-Workflows zu unterstützen.  
Mittelfristig wären Terraform-Integration, Git-Versionierung und ein Approval-Workflow mit Audit-Log denkbar.  
Langfristig bieten sich Multi-Vendor-Support, KI-gestützte Validierung, eine SaaS-Plattform und Integration in GitOps- oder \ac{OPA}-basierte Compliance-Workflows an.

\section{Schlusswort}

\textbf{XL Fabric} demonstriert, wie moderne Python-Technologien und strukturierte Validierungsprozesse die Qualität von \ac{IaC}-Workflows deutlich erhöhen können.  
Die zweistufige Validierung stellte sich als Schlüsselfaktor für Zuverlässigkeit und Datenintegrität heraus.  
Das Tool wurde termingerecht, stabil und produktionsreif umgesetzt und zeigt, wie Automatisierung und Validierung Netzwerkadministration effizienter und sicherer gestalten können.

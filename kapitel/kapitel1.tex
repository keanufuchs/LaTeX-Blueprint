\ihead{Anleitung zur Vorlage}
\chapter{Anleitung zur Verwendung dieser Vorlage}

Diese LaTeX-Vorlage dient als Basis für wissenschaftliche Hausarbeiten, Bachelor- und Masterarbeiten. Sie ist so konfiguriert, dass sie den typischen Anforderungen entspricht.

\section{Struktur der Vorlage}
Die Vorlage ist modular aufgebaut, um die Übersichtlichkeit zu wahren:

\begin{itemize}
    \item \texttt{dokument.tex}: Die Hauptdatei. Hier werden alle anderen Dateien eingebunden und globale Einstellungen (Metadaten) vorgenommen.
    \item \texttt{meta/}: Enthält Dateien für den Vorspann (Titelblatt, Abstract, Verzeichnisse).
    \item \texttt{kapitel/}: Hier liegen die eigentlichen Inhalte der Arbeit, unterteilt in Kapitel.
    \item \texttt{anhang/}: Dateien für den Anhang (zusätzliche Abbildungen, Tabellen, Code).
    \item \texttt{bib/}: Enthält die Literaturdatenbank (\texttt{literatur.bib}).
    \item \texttt{bilder/}: Speicherort für alle Grafiken und Bilder.
\end{itemize}

\section{Erste Schritte}
\begin{enumerate}
    \item Öffnen Sie die Datei \texttt{dokument.tex}.
    \item Passen Sie die Metadaten im oberen Bereich an (Autor, Titel, Matrikelnummer, etc.).
    \item Bearbeiten Sie die Dateien im Ordner \texttt{meta/} (z.B. \texttt{kurzfassung.tex}, \texttt{abkuerzungsverzeichnis.tex}).
    \item Schreiben Sie Ihren Text in den Dateien unter \texttt{kapitel/}. Sie können weitere Kapitel-Dateien anlegen und diese in \texttt{dokument.tex} mit \texttt{\textbackslash input\{kapitel/dateiname\}} einbinden.
    \item Führen Sie das Kompilieren durch (in der Regel \texttt{pdflatex} -> \texttt{biber} -> \texttt{pdflatex} -> \texttt{pdflatex}).
\end{enumerate}

\section{Vorgaben}
Beachten Sie bitte die Datei \texttt{VORGABEN.md} im Hauptverzeichnis. Sie enthält detaillierte Hinweise zur inhaltlichen Struktur der Arbeit, orientiert am Referenzprojekt.

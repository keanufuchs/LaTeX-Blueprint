\ihead{LaTeX Beispiele}
\chapter{Beispiele für LaTeX-Elemente}

Dieses Kapitel demonstriert die Verwendung häufiger Elemente in wissenschaftlichen Arbeiten.

\section{Zitate und Literaturverzeichnis}
Zitate sind essentiell für wissenschaftliches Arbeiten.
\begin{itemize}
    \item Ein Buch zitieren: \cite{mustermann2023}.
    \item Einen Artikel zitieren: \cite{doe2024}.
    \item Eine Webseite zitieren: \cite{latexproject}.
\end{itemize}
Das Literaturverzeichnis wird automatisch basierend auf den zitierten Werken erstellt.

\section{Abkürzungen (Akronyme)}
Abkürzungen sollten bei der ersten Verwendung ausgeschrieben werden. Das \texttt{acronym}-Paket übernimmt dies automatisch.
\begin{itemize}
    \item Erste Verwendung: \ac{API}.
    \item Zweite Verwendung: \ac{API}.
    \item Plural: \acp{API}.
    \item Ein weiteres Beispiel: \ac{REST}.
\end{itemize}
Die Definitionen befinden sich in \texttt{meta/abkuerzungsverzeichnis.tex}.

\section{Abbildungen}
Abbildungen werden mit der \texttt{figure}-Umgebung eingebunden. Referenzieren Sie immer auf die Abbildung im Text (siehe Abbildung \ref{fig:beispiel}).

\begin{figure}[ht]
    \centering
    \includegraphics[width=0.5\textwidth]{bilder/Beispiel-Diagramm.png}
    \caption{Ein beispielhaftes Diagramm}
    \label{fig:beispiel}
\end{figure}

\section{Tabellen}
Tabellen können einfach oder komplex sein. Tabelle \ref{tab:beispiel} zeigt ein einfaches Beispiel.

\begin{table}[ht]
    \centering
    \begin{tabular}{|l|c|r|}
        \hline
        \textbf{Links} & \textbf{Zentriert} & \textbf{Rechts} \\
        \hline
        Wert A & 123 & 10,50 \\
        Wert B & 456 & 20,00 \\
        \hline
    \end{tabular}
    \caption{Eine Beispieltabelle}
    \label{tab:beispiel}
\end{table}

\section{Quellcode}
Für Quellcode eignet sich das \texttt{listings}-Paket.

\begin{lstlisting}[language=Python, caption={Ein Python-Beispiel}, label={code:python}]
def hello_world():
    print("Hello, LaTeX Template!")
\end{lstlisting}

Referenz auf das Codebeispiel: \ref{code:python}.

\section{Mermaid-Diagramm}
Mermaid-Diagramme koennen als Quelltext dokumentiert und als Diagramm dargestellt werden.

\begin{figure}[ht]
    \centering
    \begin{tikzpicture}[
        >={Stealth[length=3mm]},
        line/.style={draw=black!70, thick, ->},
        term/.style={draw=teal!60!black, fill=teal!15, rounded corners=2pt, minimum width=3.0cm, minimum height=0.9cm, align=center, font=\small},
        process/.style={draw=blue!60!black, fill=blue!10, rounded corners=2pt, minimum width=3.4cm, minimum height=0.95cm, align=center, font=\small},
        inputstep/.style={draw=violet!60!black, fill=violet!10, rounded corners=2pt, minimum width=3.2cm, minimum height=0.95cm, align=center, font=\small},
        decision/.style={draw=orange!70!black, fill=orange!20, diamond, aspect=2.2, minimum width=3.4cm, minimum height=1.2cm, align=center, inner sep=1pt, font=\small}
    ]
        \node[term] (start) at (0,0) {Start};
        \node[decision] (check) at (0,-2.0) {Daten vorhanden?};
        \node[process] (analyze) at (0,-4.2) {Analyse starten};
        \node[inputstep] (capture) at (-4.6,-4.2) {Daten erfassen};
        \node[process] (result) at (0,-6.2) {Ergebnis dokumentieren};
        \node[term] (end) at (0,-8.1) {Ende};

        \draw[line] (start) -- (check);
        \draw[line] (check) -- node[right, font=\scriptsize] {Ja} (analyze);
        \draw[line] (check.west) -- ++(-1.2,0) |- node[pos=0.25, above, font=\scriptsize] {Nein} (capture.north);
        \draw[line] (capture.east) -- ++(1.1,0) |- (analyze.west);
        \draw[line] (analyze) -- (result);
        \draw[line] (result) -- (end);
    \end{tikzpicture}
    \caption{Gerendertes Mermaid-Beispiel im Farbschema (Flowchart)}
    \label{fig:mermaid}
\end{figure}

\begin{lstlisting}[language={}, caption={Mermaid-Beispiel (Flowchart)}, label={code:mermaid}]
flowchart TD
    A[Start] --> B{Daten vorhanden?}
    B -->|Ja| C[Analyse starten]
    B -->|Nein| D[Daten erfassen]
    D --> C
    C --> E[Ergebnis dokumentieren]
    E --> F[Ende]
\end{lstlisting}

Referenz auf das gerenderte Diagramm: \ref{fig:mermaid}. Referenz auf die Mermaid-Quelle: \ref{code:mermaid}.

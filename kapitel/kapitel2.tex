\ihead{LaTeX Beispiele}
\chapter{Beispiele innerhalb der LaTeX-Vorlage}

Dieses Kapitel zeigt typische Elemente, die in dieser LaTeX-Vorlage bereits vorbereitet sind.

\section{Zitate und Literaturverzeichnis}
Das Literaturmanagement ist in der Vorlage vorkonfiguriert. Beispielzitate:
\begin{itemize}
    \item Leitfaden zum wissenschaftlichen Schreiben mit LaTeX: \cite{mustermann2023}.
    \item Beispielartikel zur Vorlagenarbeit: \cite{doe2024}.
    \item Offizielle Projektseite von LaTeX: \cite{latexproject}.
\end{itemize}
Das Literaturverzeichnis wird aus der Datei \texttt{bib/literatur.bib} automatisch erzeugt.

\section{Abkürzungen (Akronyme)}
Abkürzungen werden zentral in \texttt{meta/abkuerzungsverzeichnis.tex} gepflegt und im Text mit dem \texttt{acronym}-Paket verwendet:
\begin{itemize}
    \item Erste Verwendung von \ac{LATEX}.
    \item Literaturarbeit mit \ac{BIBTEX}.
    \item Ausgabeformat der Arbeit: \ac{PDF}.
    \item Automatisch erzeugte Verzeichnisse: \ac{TOC}, \ac{LOF}, \ac{LOT}, \ac{LOL}.
\end{itemize}

\section{Abbildungen}
Abbildungen werden in der \texttt{figure}-Umgebung eingefügt. Ein Beispiel zeigt Abbildung \ref{fig:beispiel}.

\begin{figure}[H]
    \centering
    \includegraphics[width=0.5\textwidth]{bilder/Beispiel-Diagramm.png}
    \caption{Beispielabbildung in der LaTeX-Vorlage}
    \label{fig:beispiel}
\end{figure}

\section{Tabellen}
Tabelle \ref{tab:beispiel} zeigt eine einfache tabellarische Darstellung.

\begin{table}[H]
    \centering
    \begin{tabular}{|l|c|r|}
        \hline
        \textbf{Element} & \textbf{Anzahl} & \textbf{Status} \\
        \hline
        Kapiteldateien & 2 & aktiv \\
        Anhangsdateien & 7 & vorbereitet \\
        \hline
    \end{tabular}
    \caption{Einfache Beispieltabelle}
    \label{tab:beispiel}
\end{table}

Tabelle \ref{tabelle:nutzwertanalyse_diagramme} zeigt ein komplexeres Beispiel als Nutzwertanalyse für Diagrammoptionen innerhalb dieser LaTeX-Vorlage.

\begin{table}[H]
    \begin{center}
    \scriptsize
    \begin{tabular}{|c|c|c|c|c|c|c|c|}
            \hline
            \multirow{2}{*}{Kriterien} & \multirow{2}{*}{Gewichtung} & \multicolumn{2}{c|}{\makecell{PNG/JPG}} & \multicolumn{2}{c|}{\makecell{TikZ}} & \multicolumn{2}{c|}{\makecell{Mermaid \\ (extern)}} \\
            \cline{3-8}
             &  & Bew. & Ges. & Bew. & Ges. & Bew. & Ges. \\
            \hline
            \makecell{Einrichtungsaufwand} & 20\% & 5 & 1,00 & 3 & 0,60 & 4 & 0,80 \\
            \makecell{Typografische \\ Konsistenz} & 25\% & 2 & 0,50 & 5 & 1,25 & 4 & 1,00 \\
            \makecell{Anpassbarkeit im \\ Quelltext} & 25\% & 1 & 0,25 & 5 & 1,25 & 4 & 1,00 \\
            \makecell{Kompilier- \\ geschwindigkeit} & 15\% & 5 & 0,75 & 2 & 0,30 & 3 & 0,45 \\
            \makecell{Wiederverwendbarkeit} & 15\% & 3 & 0,45 & 4 & 0,60 & 5 & 0,75 \\
            \hline
            GESAMT & 100\% & - & 2,95 & - & 4,00 & - & 4,00 \\
            \hline
    \end{tabular}
    \end{center}
    \caption{Nutzwertanalyse für Diagrammoptionen in der LaTeX-Vorlage}
    \label{tabelle:nutzwertanalyse_diagramme}
\end{table}

\section{Quellcode}
Für Quellcodebeispiele ist das \texttt{listings}-Paket eingebunden.

\begin{lstlisting}[language=Python, caption={Beispielhafter Build-Ablauf fuer die LaTeX-Vorlage}, label={code:python}]
import subprocess

steps = [
    "pdflatex dokument.tex",
    "bibtex dokument",
    "pdflatex dokument.tex",
    "pdflatex dokument.tex",
]

for step in steps:
    subprocess.run(step, shell=True, check=True)
\end{lstlisting}

Referenz auf das Codebeispiel: \ref{code:python}.

\section{Mermaid-Diagramm}
Mermaid-Diagramme koennen als Quelltext dokumentiert und als Abbildungen inhaltlich gespiegelt werden.

\subsection{Flowchart}

\begin{figure}[H]
    \centering
    \begin{tikzpicture}[
        >={Stealth[length=3mm]},
        line/.style={draw=black!70, thick, ->},
        term/.style={draw=teal!60!black, fill=teal!15, rounded corners=2pt, minimum width=3.0cm, minimum height=0.9cm, align=center, font=\small},
        process/.style={draw=blue!60!black, fill=blue!10, rounded corners=2pt, minimum width=3.6cm, minimum height=0.95cm, align=center, font=\small},
        inputstep/.style={draw=violet!60!black, fill=violet!10, rounded corners=2pt, minimum width=3.4cm, minimum height=0.95cm, align=center, font=\small},
        decision/.style={draw=orange!70!black, fill=orange!20, diamond, aspect=2.2, minimum width=3.6cm, minimum height=1.2cm, align=center, inner sep=1pt, font=\small}
    ]
        \node[term] (start) at (0,0) {Start};
        \node[decision] (check) at (0,-2.0) {Quellen aktuell?};
        \node[process] (compile) at (0,-4.2) {PDF kompilieren};
        \node[inputstep] (update) at (-4.9,-4.2) {Bib-Datei ergaenzen};
        \node[process] (review) at (0,-6.2) {Ergebnis pruefen};
        \node[term] (end) at (0,-8.1) {Ende};

        \draw[line] (start) -- (check);
        \draw[line] (check) -- node[right, font=\scriptsize] {Ja} (compile);
        \draw[line] (check.west) -- ++(-1.2,0) -| node[pos=0.25, above, font=\scriptsize] {Nein} (update.north);
        \draw[line] (update.east) -- ++(1.1,0) |- (compile.west);
        \draw[line] (compile) -- (review);
        \draw[line] (review) -- (end);
    \end{tikzpicture}
    \caption{Gerendertes Mermaid-Beispiel (Flowchart fuer den Build-Prozess)}
    \label{fig:mermaid}
\end{figure}

\begin{lstlisting}[language={}, caption={Mermaid-Beispiel (Flowchart fuer den Build-Prozess)}, label={code:mermaid}]
flowchart TD
    A[Start] --> B{Quellen aktuell?}
    B -->|Ja| C[PDF kompilieren]
    B -->|Nein| D[Bib-Datei ergaenzen]
    D --> C
    C --> E[Ergebnis pruefen]
    E --> F[Ende]
\end{lstlisting}

Referenz auf das gerenderte Diagramm: \ref{fig:mermaid}. Referenz auf die Mermaid-Quelle: \ref{code:mermaid}.

\subsection{Gantt-Diagramm mit Meilensteinen}

\begin{figure}[H]
    \centering
    \begin{ganttchart}[
        hgrid,
        vgrid,
        x unit=0.90cm,
        y unit chart=0.70cm,
        title/.append style={fill=teal!15, draw=teal!70!black, rounded corners=1pt},
        title label font=\scriptsize\bfseries\color{teal!80!black},
        group/.append style={fill=teal!20, draw=teal!70!black, rounded corners=1pt},
        group label font=\scriptsize\bfseries\color{teal!80!black},
        bar/.append style={fill=blue!12, draw=blue!65!black, rounded corners=1pt},
        bar label font=\scriptsize,
        milestone/.append style={fill=orange!70, draw=orange!90!black},
        milestone label font=\scriptsize\bfseries\color{orange!90!black}
    ]{1}{10}
        \gantttitle{Projektzeitraum (Wochen)}{10} \\
        \gantttitlelist{1,...,10}{1} \\
        \ganttgroup{Planung}{1}{2} \\
        \ganttbar[bar/.append style={fill=violet!12, draw=violet!70!black}]{Struktur festlegen}{1}{2} \\
        \ganttgroup{Schreibphase}{3}{8} \\
        \ganttbar{Kapitelentwurf}{3}{6} \\
        \ganttbar{Abbildungen und Tabellen}{5}{7} \\
        \ganttbar[bar/.append style={fill=blue!20, draw=blue!75!black}]{Quellen einarbeiten}{7}{8} \\
        \ganttgroup{Finalisierung}{9}{10} \\
        \ganttbar[bar/.append style={fill=teal!12, draw=teal!70!black}]{Layout- und Sprachcheck}{9}{10} \\
        \ganttmilestone{MS: Rohfassung}{7} \\
        \ganttmilestone{MS: Abgabe-PDF}{10}
    \end{ganttchart}
    \caption{Gerendertes Mermaid-Beispiel im passenden Farbschema (Gantt mit Meilensteinen)}
    \label{fig:mermaid-gantt}
\end{figure}

\begin{lstlisting}[language={}, caption={Mermaid-Beispiel (Gantt mit Meilensteinen)}, label={code:mermaid-gantt}]
gantt
    title Schreib- und Build-Plan der LaTeX-Vorlage
    dateFormat  YYYY-MM-DD
    axisFormat  %W

    section Planung
    Struktur festlegen        :a1, 2026-03-03, 14d

    section Schreibphase
    Kapitelentwurf            :a2, after a1, 28d
    Abbildungen und Tabellen  :a3, after a2, 14d
    Quellen einarbeiten       :a4, after a3, 7d

    section Meilensteine
    Rohfassung                :milestone, m1, 2026-04-21, 0d
    Abgabe-PDF                :milestone, m2, 2026-05-05, 0d
\end{lstlisting}

Referenz auf das gerenderte Gantt-Diagramm: \ref{fig:mermaid-gantt}. Referenz auf die Mermaid-Quelle: \ref{code:mermaid-gantt}.

\ihead{LaTeX Beispiele}
\chapter{Beispiele für LaTeX-Elemente}

Dieses Kapitel demonstriert die Verwendung häufiger Elemente in wissenschaftlichen Arbeiten.

\section{Zitate und Literaturverzeichnis}
Zitate sind essentiell für wissenschaftliches Arbeiten.
\begin{itemize}
    \item Ein Buch zitieren: \cite{mustermann2023}.
    \item Einen Artikel zitieren: \cite{doe2024}.
    \item Eine Webseite zitieren: \cite{latexproject}.
\end{itemize}
Das Literaturverzeichnis wird automatisch basierend auf den zitierten Werken erstellt.

\section{Abkürzungen (Akronyme)}
Abkürzungen sollten bei der ersten Verwendung ausgeschrieben werden. Das \texttt{acronym}-Paket übernimmt dies automatisch.
\begin{itemize}
    \item Erste Verwendung: \ac{API}.
    \item Zweite Verwendung: \ac{API}.
    \item Plural: \acp{API}.
    \item Ein weiteres Beispiel: \ac{REST}.
\end{itemize}
Die Definitionen befinden sich in \texttt{meta/abkuerzungsverzeichnis.tex}.

\section{Abbildungen}
Abbildungen werden mit der \texttt{figure}-Umgebung eingebunden. Referenzieren Sie immer auf die Abbildung im Text (siehe Abbildung \ref{fig:beispiel}).

\begin{figure}[ht]
    \centering
    \includegraphics[width=0.5\textwidth]{bilder/Beispiel-Diagramm.png}
    \caption{Ein beispielhaftes Diagramm}
    \label{fig:beispiel}
\end{figure}

\section{Tabellen}
Tabellen können einfach oder komplex sein. Tabelle \ref{tab:beispiel} zeigt ein einfaches Beispiel.

\begin{table}[ht]
    \centering
    \begin{tabular}{|l|c|r|}
        \hline
        \textbf{Links} & \textbf{Zentriert} & \textbf{Rechts} \\
        \hline
        Wert A & 123 & 10,50 \\
        Wert B & 456 & 20,00 \\
        \hline
    \end{tabular}
    \caption{Eine Beispieltabelle}
    \label{tab:beispiel}
\end{table}

\section{Quellcode}
Für Quellcode eignet sich das \texttt{listings}-Paket.

\begin{lstlisting}[language=Python, caption={Ein Python-Beispiel}, label={code:python}]
def hello_world():
    print("Hello, LaTeX Template!")
\end{lstlisting}

Referenz auf das Codebeispiel: \ref{code:python}.

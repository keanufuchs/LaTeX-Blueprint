\chapter*{Hinweise}
\ihead{Hinweise}

\addcontentsline{toc}{chapter}{Hinweise} % Fügt die Überschrift in das Inhaltsverzeichnis ein, ohne zu nummerieren.

Diese Arbeit beschreibt das Projekt \textbf{XL Fabric}, ein Python-Tool zur Validierung und Konvertierung von Cisco \ac{ACI} Konfigurationsdaten. Das Tool liest Day-0-Konfigurationen aus Excel-Dokumenten ein und transformiert diese in Terraform-kompatible \acf{YAML}-Dateien. Der Schwerpunkt des Projekts liegt auf der zweistufigen Validierung mittels Pydantic: Zunächst werden die Excel-Daten auf Vollständigkeit, korrekte Formate und Konsistenz geprüft (Input-Validierung), anschließend wird die Terraform-Kompatibilität und Datenintegrität vor dem Export sichergestellt (Output-Validierung).

\textbf{Wichtiger Hinweis:} XL Fabric konzentriert sich ausschließlich auf die Terraform-Ausgabe. Obwohl im Code Ansible-Funktionalität vorbereitet ist (\acf{CLI}-Flag \texttt{--ansible}), wurde diese nicht implementiert oder weiterentwickelt. Die gesamte Dokumentation und alle Tests beziehen sich daher ausschließlich auf den Terraform-Workflow. Die tatsächliche Anwendung der Konfiguration auf die \ac{ACI} Fabric mittels Terraform, die Entwicklung oder Anpassung der verwendeten Terraform-Module sowie die Verwaltung oder Überwachung von \ac{ACI}-Infrastrukturen sind explizit nicht Teil dieses Projekts.

In dieser Arbeit werden häufig Abkürzungen verwendet. Bei der ersten Verwendung wird die vollständige Bezeichnung angegeben, gefolgt von der Abkürzung in Klammern. Ein vollständiges Abkürzungsverzeichnis findet sich zu Beginn dieser Dokumentation. Code-Beispiele werden in Verbatim-Umgebungen dargestellt und verwenden Python 3.11 Syntax. Die verwendeten Bibliotheken und ihre Versionen sind im Kapitel \enquote{Vorgehensweise und Methodik} dokumentiert.